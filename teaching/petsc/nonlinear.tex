% -*- latex
%%%%%%%%%%%%%%%%%%%%%%%%%%%%%%%%%%%%%%%%%%%%%%%%%%%%%%%%%%%%%%%%
%%%%
%%%% This TeX file is part of the tutorial
%%%% `Introduction to the PETSc library'
%%%% Victor Eijkhout, eijkhout@tacc.utexas.edu
%%%% copyright Victor Eijkhout 2012-2020
%%%%
%%%%%%%%%%%%%%%%%%%%%%%%%%%%%%%%%%%%%%%%%%%%%%%%%%%%%%%%%%%%%%%%

\sectionframe{\texttt{SNES}: Nonlinear solvers}

\frame{\frametitle{Nonlinear problems}

Basic equation
\[ f(u) = 0 \]
where $u$ can be big, for instance nonlinear PDE.

Typical solution method:
\[ u_{n+1} = u_n- J(u_n)\inv f(u_n) \]
Newton iteration.

Needed: function and Jacobian.
}

\frame[containsverbatim]{\frametitle{Basic SNES usage}

User supplies function and Jacobian:
\begin{verbatim}
SNES           snes;

SNESCreate(PETSC_COMM_WORLD,&snes)
SNESSetType(snes,type);
SNESSetFromOptions(snes);
SNESDestroy(SNES snes);
\end{verbatim}

where \n{type}: 
\begin{itemize}
\item \n{SNESLS} Newton with line search
\item \n{SNESTR} Newton with trust region
\item several specialized ones
\end{itemize}
}

\begin{frame}[containsverbatim]{SNES specification: function evaluation}
\begin{lstlisting}
PetscErrorCode (*FunctionEvaluation)(SNES,Vec,Vec,void*);
VecCreate(PETSC_COMM_WORLD,&r);
SNESSetFunction(snes,r,FunctionEvaluation,*ctx);
\end{lstlisting}
\end{frame}

\begin{frame}[containsverbatim]{SNES specification: jacobian evaluation}
\begin{lstlisting}
PetscErrorCode (*FormJacobian)(SNES,Vec,Mat,Mat,void*);
MatCreate(PETSC_COMM_WORLD,&J);
SNESSetJacobian(snes,J,J,FormJacobian,*ctx);
\end{lstlisting}
\end{frame}

\begin{frame}[containsverbatim]{SNES solution}
\begin{lstlisting}
SNESSolve(snes,/* rhs= */ PETSC_NULL,x)
SNESGetConvergedReason(snes,&reason)
SNESGetIterationNumber(snes,&its)
\end{lstlisting}
\end{frame}

\frame[containsverbatim]{\frametitle{Example: two-variable problem}
Define a context
\begin{verbatim}
typedef struct {
  Vec xloc,rloc; VecScatter scatter; } AppCtx;

/* User context */
AppCtx         user;

/* Work vectors in the user context */
VecCreateSeq(PETSC_COMM_SELF,2,&user.xloc);
VecDuplicate(user.xloc,&user.rloc);

/* Create the scatter between the global and local x */
ISCreateStride(MPI_COMM_SELF,2,0,1,&idx);
VecScatterCreate(x,idx,user.xloc,idx,&user.scatter);
\end{verbatim}
}

\frame[containsverbatim]{
In the user function:
\begin{verbatim}
PetscErrorCode FormFunction
    (SNES snes,Vec x,Vec f,void *ctx)
{
  VecScatterBegin(user->scatter,
    x,user->xloc,INSERT_VALUES,SCATTER_FORWARD); // & End
  VecGetArray(xloc,&xx);CHKERRQ(ierr);
  VecSetValue
    (f,0,/* something with xx[0]) & xx[1] */,
     INSERT_VALUES);
  VecRestoreArray(x,&xx);
  PetscFunctionReturn(0);
}
\end{verbatim}
}

\frame[containsverbatim]{\frametitle{Jacobian calculation through
    finite differences}
Jacobian calculation is difficult. It can be approximated through
finite differences:
\[ J(u)v \approx \frac{f(u+hv)-f(u)}{h} \]
\begin{verbatim}
MatCreateSNESMF(snes,&J);
SNESSetJacobian
 (snes,J,J,MatMFFDComputeJacobian,(void*)&user);
\end{verbatim}
}

\frame[containsverbatim]{\frametitle{Further possibilities}

\n{SNESSetTolerances(SNES snes,double atol,double rtol,double stol,
int its,int fcts);}

convergence test and monitoring, specific options for line search and
trust region

adaptive converngence: \n{-snes_ksp_ew_conv} (Eisenstat Walker)
}


\frame[containsverbatim]{\frametitle{Solve customization}
\begin{verbatim}
SNESSetType(snes,SNESTR); /* newton with trust region */
SNESGetKSP(snes,&ksp)
KSPGetPC(ksp,&pc)
PCSetType(pc,PCNONE)
KSPSetTolerances(ksp,1.e-4,PETSC_DEFAULT,PETSC_DEFAULT,20)
\end{verbatim}
}

\endinput

PetscMPIInt    size,rank;
PetscScalar    pfive = .5,*xx;
PetscTruth     flg;
AppCtx         user;         /* user-defined work context */
IS             isglobal,islocal;

/* ------------------------------------------------------------------- */
#undef __FUNCT__
#define __FUNCT__ "FormJacobian2"


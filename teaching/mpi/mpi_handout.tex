% -*- latex -*-
%%%%%%%%%%%%%%%%%%%%%%%%%%%%%%%%%%%%%%%%%%%%%%%%%%%%%%%%%%%%%%%%
%%%%%%%%%%%%%%%%%%%%%%%%%%%%%%%%%%%%%%%%%%%%%%%%%%%%%%%%%%%%%%%%
%%%%
%%%% This text file is part of the source of 
%%%% `Parallel Computing'
%%%% by Victor Eijkhout, copyright 2012-2020
%%%%
%%%% tacc_handout.tex : practical details 
%%%%
%%%%%%%%%%%%%%%%%%%%%%%%%%%%%%%%%%%%%%%%%%%%%%%%%%%%%%%%%%%%%%%%
%%%%%%%%%%%%%%%%%%%%%%%%%%%%%%%%%%%%%%%%%%%%%%%%%%%%%%%%%%%%%%%%

\documentclass[11pt]{artikel3}

\usepackage[pdftex]{hyperref}
\usepackage{pslatex}

\begin{document}
%\title{Guide to the MPI labs\\ 2020 Summer Institute}
\title{Introduction to Advaned MPI features\\ PEARC 2020}
\author{}\date{}
\maketitle

\section{Access to a cluster}

You need to connect to a designated machine for doing exercises.

\begin{itemize}
\item Log in to the designated machine: (Stampede2/Frontera)
\begin{verbatim}
ssh -X -l yourname stampede2.tacc.utexas.edu
# or
ssh -X -l yourname frontera.tacc.utexas.edu
\end{verbatim}
(If your name is the same at TACC as on your local machine,
you can leave out the \texttt{-l yourname} option.)
\item When prompted, use reservation
  %%%% SPECIFICS
  \texttt{PEARC20\_MPI}.
\item If you already have an account on this machine,
  you may be prompted for which project to use.
  Your own or the class project
  %%%% SPECIFICS
  %%\texttt{TRAINING-HPC}
  \texttt{Frontera-Training}
  will both work.
\end{itemize}

\section{Course materials}

Course materials consist of pdf files and
and exercise files.
You probably want to open the pdf files
on your local machine, so get them from:
%%%% SPECIFICS
\begin{verbatim}
https://tinyurl.com/tacc-2020-pearc
\end{verbatim}

For doing the exercises you can find the same
material on the cluster:
\begin{itemize}
\item Unpack the course materials in your directory:
%%%% SPECIFICS
\begin{verbatim}
tar fxz ~train00/mpithree_course_2020.tgz
\end{verbatim}
\item Go into the course directory:
%%%% SPECIFICS
\begin{verbatim}
cd mpithree_course
\end{verbatim}
\item You will find pdf files of the course and of a full textbook.
  It's a good idea to keep these open for reference.
\item You will also find exercise directories for the languages~C, C++,
  Fortran, Fortran2008, and Python. Pick your favourite.
\item Fortran: it is advisable to use the \texttt{f08} directory.
\item For C++ there is also the MPL package. To use this:
\begin{verbatim}
export TACC_MPL_INC=/work/00434/eijkhout/mpl/mpl
\end{verbatim}
\end{itemize}

\section{Doing exercises}

Review your slides about good citizenship at TACC:
do not run parallel programs on the login nodes. Instead,
follow these instructions:

\begin{itemize}
\item Start an interactive session, ideally on 2 nodes, for the
  duration of this course:
\begin{verbatim}
idev -N 2 -n 20 -t 6:0:0
\end{verbatim}
That is: two nodes with 20 processes total, for 6 hours.

(Given 2 nodes, you can request up to 96 processes on Stampede2
and 116 on Frontera, but that will make a mess of your screen output.
Keep it small while you're learning.)
\item The exercises will be named in the course. For instance, there
  will be a `hello' exercise. The setup for this exercise will be in
  the file \texttt{hello.c/cxx/F90/py}. To compile the exercise, do
\begin{verbatim}
make hello
\end{verbatim}
This will give you a program that you can run in parallel with \texttt{ibrun}:
\begin{verbatim}
ibrun hello
\end{verbatim}
\end{itemize}

\end{document}

% -*- latex -*-
%%%%%%%%%%%%%%%%%%%%%%%%%%%%%%%%%%%%%%%%%%%%%%%%%%%%%%%%%%%%%%%%
%%%%%%%%%%%%%%%%%%%%%%%%%%%%%%%%%%%%%%%%%%%%%%%%%%%%%%%%%%%%%%%%
%%%%
%%%% This text file is part of the source of 
%%%% `Parallel Programming in MPI and OpenMP'
%%%% by Victor Eijkhout, copyright 2012-2022
%%%%
%%%% omp-task.tex : openmp tasking
%%%%
%%%%%%%%%%%%%%%%%%%%%%%%%%%%%%%%%%%%%%%%%%%%%%%%%%%%%%%%%%%%%%%%
%%%%%%%%%%%%%%%%%%%%%%%%%%%%%%%%%%%%%%%%%%%%%%%%%%%%%%%%%%%%%%%%

\label{sec:omp:task}
\index{OpenMP!tasks|(}

Tasks are a mechanism that OpenMP uses under the cover:
if you specify something as being parallel, OpenMP will create
a `block of work': a~section of code plus the data environment
in which it occurred. This block is set aside for execution at some later point.

Let's look at a simple example using the \indexpragma{task} directive.
\begin{quotation}
  \begin{tabular}{|ll|}
    \hline
    Code&Execution\\
    \hline
    \n{\ x = f();}& the variable \n{x} gets a value\\
    \n{#pragma omp task}&a task is created with the current value of \texttt{x}\\
    \n{\ \{ y = g(x); \}}&\\
    \n{\ z = h();}& the variable \n{z} gets a value\\
    \hline
  \end{tabular}
\end{quotation}
The thread that executes this code segment creates a task,
which will later be executed, probably by a different thread.
The exact timing of the execution of the task is up to a \indextermbus{task}{scheduler},
which operates invisible to the user.

The task mechanism allows you to do things that are hard or impossible
with the loop and section constructs. For instance, a \indexterm{while
  loop} traversing a \indexterm{linked list} can be implemented with tasks:
\begin{quotation}
  \begin{tabular}{|ll|}
    \hline
    Code&Execution\\
    \hline
    \n{p = head_of_list();}& one thread traverses the list\\
    \n{while (!end_of_list(p)) \{}\\
    \n{#pragma omp task}& a task is created,\\
    \n{\ process( p );}& one for each element\\
    \n{\ p = next_element(p);}& the generating thread goes on without waiting\\
    \n{\}}&the tasks are executed while \\
          &more are being generated.
  \end{tabular}
\end{quotation}

The way tasks and threads interact is different from the worksharing
constructs you've seen so far. Typically, one thread will generate the
tasks, adding them to a queue, from which all threads can take and
execute them. This leads to the following idiom:
\begin{lstlisting}
#pragma omp parallel
#pragma omp single
{
  ...
#pragma omp task
  { ... }
  ...
}
\end{lstlisting}
\begin{enumerate}
\item A parallel region creates a team of threads;
\item a single thread then creates the tasks, adding them to a queue
  that belongs to the team,
\item and all the threads in that team (possibly including the one
  that generated the tasks) 
\end{enumerate}

With tasks it becomes possible to parallelize processes that did not
fit the earlier OpenMP constructs. For instance, if a certain
operation needs to be applied to all elements of a linked list, you
can have one thread go down the list, generating a task for each
element of the list.

Another concept that was hard to parallelize earlier is the `while
loop'. This does not fit the requirement for OpenMP parallel loops
that the loop bound needs to be known before the loop executes.

\begin{exercise}
  \label{ex:taskfactor}
  Use tasks to find the smallest factor of a large
  number (using $2999\cdot 3001$ as test case): generate a task for each
  trial factor. Start with this code:
\begin{lstlisting}
  int factor=0;
#pragma omp parallel
#pragma omp single
  for (int f=2; f<4000; f++) {
    { // see if `f' is a factor
      if (N%f==0) { // found factor!
        factor = f;
      }
    }
    if (factor>0)
      break;
  }
  if (factor>0)
    printf("Found a factor: %d\n",factor);
\end{lstlisting}
  \begin{itemize}
  \item Turn the factor finding block into a task.
  \item Run your program a number of times:
\begin{verbatim}
for i in `seq 1 1000` ; do ./taskfactor ; done | grep -v 2999
\end{verbatim}
    Does it find the wrong factor? Why? Try to fix this.
\item Once a factor has been found, you should stop generating
    tasks.
    Let tasks that should not have been generated, meaning that they
    test a candidate larger than the factor found, print out a message.
  \end{itemize}
\end{exercise}


\Level 0 {Task data}
\index{OpenMP!tasks!data|(}

Treatment of data in a task is somewhat subtle. The basic problem is
that a task gets created at one time, and executed at another. Thus,
if shared data is accessed, does the task see the value at creation
time or at execution time? In fact, both possibilities make sense
depending on the application, so we need to discuss the rules when
which possibility applies.

The first rule is that shared data is shared in the task, but private
data becomes \indexclause{firstprivate}. To see the distinction, consider two
code fragments. In the first example:
\begin{lstlisting}
int count = 100;
#pragma omp parallel
#pragma omp single
{
  while (count>0) {
#pragma omp task
    {
      int countcopy = count;
      if (count==50) {
        sleep(1);
        printf("%d,%d\n",count,countcopy);
      } // end if
    }   // end task
    count--;
  }     // end while
}       // end single
\end{lstlisting}
the variable \n{count} is declared outside the
parallel region and is therefore shared. When the print statement is
executed, all tasks will have been generated, and so \n{count} will be
zero. Thus, the output will likely be \n{0,50}.

In the second example:
\begin{lstlisting}
#pragma omp parallel
#pragma omp single
{
  int count = 100;
  while (count>0) {
#pragma omp task
    {
      int countcopy = count;
      if (count==50) {
        sleep(1);
        printf("%d,%d\n",count,countcopy);
      } // end if
    }   // end task
    count--;
  }     // end while
}       // end single
\end{lstlisting}
the \n{count} variable is private to the thread creating the tasks,
and so it will be \n{firstprivate} in the task, preserving the value
that was current when the task was created.

\index{OpenMP!tasks!data|)}

\Level 0 {Task synchronization}
\index{OpenMP!tasks!synchronization|(}

Even though the above segment looks like a linear set of statements,
it is impossible to say when
the code after the \n{task} directive will be executed.
This means that the following code is incorrect:
\begin{lstlisting}
  x = f();
#pragma omp task
  { y = g(x); }
  z = h(y);  
\end{lstlisting}
Explanation: when the statement computing \n{z} is executed, the task
computing~\n{y} has only been scheduled;
it has not necessarily been executed yet.

In order to have a guarantee that a task is finished,
you need the \indexpragma{taskwait} directive.
The following creates two tasks, which can be executed
in parallel, and then waits for the results:
\begin{quotation}
  \begin{tabular}{|ll|}
    \hline
    Code&Execution\\
    \hline
    \n{\ x = f();}& the variable \n{x} gets a value\\
    \n{#pragma omp task}&\multirow{4}{*}{two tasks are created with the current value of \texttt{x}}\\
    \n{\ \{ y1 = g1(x); \}}&\\
    \n{#pragma omp task}&\\
    \n{\ \{ y2 = g2(x); \}}&\\
    \n{#pragma omp taskwait}& the thread waits until the tasks are finished\\
    \n{\ z = h(y1)+h(y2);}& the variable \n{z} is computed using the task results\\
    \hline
  \end{tabular}
\end{quotation}

The \indexpragma{task} pragma is followed by a structured block.
Each time the structured block is encountered, a new task is generated.
On the other hand \indexpragma{taskwait} is a standalone directive; 
the code that follows is just code, it is not a structured block belonging
to the directive.

Another aspect of the distinction between generating tasks and executing them:
usually the tasks are generated by one thread, but executed by many threads.
Thus, the typical idiom is:
\begin{lstlisting}
#pragma omp parallel
#pragma omp single
{
  // code that generates tasks
}  
\end{lstlisting}

This makes it possible to execute loops in parallel
that do not have the right kind of iteration structure
for a \n{omp parallel for}. As an example, you
could traverse and process a linked list:
\begin{lstlisting}
#pragma omp parallel
#pragma omp single
{
  while (!tail(p)) {
    p = p->next();
#pragma omp task
    process(p)
  }
#pragma omp taskwait
}
\end{lstlisting}
One task traverses the linked list creating an independent
task for each element in the list. These tasks are then
executed in parallel; their assignment to threads
is done by the task scheduler.

You can indicate task dependencies in several ways:
\begin{enumerate}
\item Using the `task wait' directive you can explicitly indicate
  the \emph{join}\index{fork/join model} of the
  \emph{forked} tasks. The instruction after the wait directive
  will therefore be dependent on the spawned tasks.
\item The \indexpragma{taskgroup} directive, followed
  by a structured block, ensures completion of all tasks
  created in the block, even if recursively created.
\item Each OpenMP task can have a \indexclause{depend}
  clause, indicating what \indexterm{data dependency} of the task.
  By indicating what data is produced or absorbed by the tasks,
  the scheduler can construct the dependency graph for you.
\end{enumerate}

Another mechanism for dealing with tasks is the
\indexpragma{taskgroup}: a task group is a code block that can
contain \indexpragma{task} directives; all these tasks need to be
finished before any statement after the block is executed.

A task group is somewhat similar to having a \indexpragma{taskwait}
directive after the block. The big difference is that that
\n{taskwait} directive does not wait for tasks that are recursively
generated, while a \n{taskgroup} does.

\index{OpenMP!tasks!synchronization|)}

\Level 0 {Task dependencies}
\index{OpenMP!tasks!dependencies|(}

It is possible to put a partial ordering on
tasks through use of the \indexclause{depend} clause. For example, in
\begin{lstlisting}
#pragma omp task
  x = f()
#pragma omp task
  y = g(x)
\end{lstlisting}
it is conceivable that the second task is executed before the first,
possibly leading to an incorrect result. This is remedied by specifying:
\begin{lstlisting}
#pragma omp task depend(out:x)
  x = f()
#pragma omp task depend(in:x)
  y = g(x)
\end{lstlisting}

\begin{exercise}
Consider the following code:
\begin{lstlisting}
for i in [1:N]:
    x[0,i] = some_function_of(i)
    x[i,0] = some_function_of(i)

for i in [1:N]:
    for j in [1:N]:
        x[i,j] = x[i-1,j]+x[i,j-1]
\end{lstlisting}
\begin{itemize}
\item Observe that the second loop nest is not amenable to OpenMP loop
  parallelism.
\item Can you think of a way to realize the computation with OpenMP
  loop parallelism? Hint: you need to rewrite the code so that the
  same operations are done in a different order.
\item Use tasks with dependencies to make this code parallel without
  any rewriting: the only change is to add OpenMP directives.
\end{itemize}
\end{exercise}

Tasks dependencies are used to indicated how two uses of one data item
relate to each other. Since either use can be a read or a write,
there are four types of dependencies.

\begin{description}
\item[RaW (Read after Write)] The second task reads an item that the
  first task writes. The second task has to be executed after the
  first:
\begin{lstlisting}
... omp task depend(OUT:x)
  foo(x)
... omp task depend( IN:x)
  foo(x)
\end{lstlisting}
\item[WaR (Write after Read)] The first task reads and item, and the
  second task overwrites it. The second task has to be executed second
  to prevent overwriting the initial value:
\begin{lstlisting}
... omp task depend( IN:x)
  foo(x)
... omp task depend(OUT:x)
  foo(x)
\end{lstlisting}
\item[WaW (Write after Write)] Both tasks set the same variable. Since
  the variable can be used by an intermediate task, the two writes
  have to be executed in this order.
\begin{lstlisting}
... omp task depend(OUT:x)
  foo(x)
... omp task depend(OUT:x)
  foo(x)
\end{lstlisting}
\item[RaR (Read after Read)] Both tasks read a variable. Since neither
  tasks has an `out' declaration, they can run in either order.
\begin{lstlisting}
... omp task depend(IN:x)
  foo(x)
... omp task depend(IN:x)
  foo(x)
\end{lstlisting}
\end{description}
\index{OpenMP!tasks!dependencies|)}

\Level 0 {Task reduction}

The \indexclause{reduction} only pertains to ordinary parallel loops,
not to \indexpragma{taskgroup} loops of tasks.
To do a reduction over computations in tasks you need the
\indexclausedef{task_reduction} clause
(a~\ompstandard{5.0} feature):
\begin{lstlisting}
#pragma omp taskgroup task_reduction(+:sum)
\end{lstlisting}
The task group can contain both task that contribute to the reduction,
and ones that don't. The former type needs a clause \indexclausedef{in_reduction}:
\begin{lstlisting}
#pragma omp task in_reduction(+:sum)
\end{lstlisting}

As an example, here the sum $\sum_{i=1}^{100} i$ is computed with tasks:
%
\cverbatimsnippet{omptaskreduct}

\Level 0 {More}

\Level 1 {Scheduling points}

Normally, a task stays tied to the thread that first executes it.
However, at a \indextermbus{task}{scheduling point} the thread may
switch to the execution of another task created by the same team.
\begin{itemize}
\item There is a scheduling point after explicit task creation. This
  means that, in the above examples, the thread creating the tasks can
  also participate in executing them.
\item There is a scheduling point at \indexpragma{taskwait} and \indexpragma{taskyield}.
\end{itemize}

On the other hand a task created with them \indexclause{untied} clause
on the task pragma is never tied to one thread. This means that after
suspension at a scheduling point any thread can resume execution of
the task.
If you do this, beware
that the value of a thread-id does not stay fixed. Also locks become a problem.

Example: if a thread is waiting for a lock, with a scheduling point it
can suspend the task and work on another task.
\begin{lstlisting}
while (!omp_test_lock(lock))
#pragma omp taskyield
  ;
\end{lstlisting}

\Level 1 {Hints for performance improvement}
\label{sec:omp-task-hint}

If a task involves only a small amount of work,
the scheduling overhead may negate any performance gain.
There are two ways of executing the task code directly:
\begin{itemize}
\item The \indexompshow{if} clause will only create a task
  if the test is true:
\begin{lstlisting}
#pragma omp task if (n>100)
  f(n)
\end{lstlisting}
\item The \indexompshow{if} clause may still lead to recursively
  generated tasks. On the other hand, \indexompshow{final}
  will execute the code, and will also skip any recursively created tasks:
\begin{lstlisting}
#pragma omp task final(level<3)
\end{lstlisting}
\end{itemize}

If you want to indicate that certain tasks are more important
than others, use the \indexompshow{priority} clause:
\begin{lstlisting}
#pragma omp task priority(5)
\end{lstlisting}
where the priority is any non-negative scalar
less than \indexompdef{OMP_MAX_TASK_PRIORITY}.

\Level 1 {Task cancelling}

It is possible (in \ompstandard{4.0}) to cancel
tasks. This is useful when tasks are used to perform a search: the
task that finds the result first can cancel any outstanding search
tasks.
See section~\ref{sec:omp-cancel} for details.

\begin{exercise}
  Modify the prime finding example to use \indexompshow{cancel}.
\end{exercise}

\Level 0 {Examples}

\Level 1 {Fibonacci}

As an example of the use of tasks, consider computing an array of Fibonacci values:
%
\cverbatimsnippet[examples/omp/c/taskgroup0.c]{fiboarray}
%
If you simply turn each calculation into a task, results will be
unpredictable (confirm this!) since tasks can be executed in any sequence.
To solve this, we put dependencies on the tasks:
%
\cverbatimsnippet[examples/omp/c/taskgroup2.c]{fibotaskdepend}

\Level 1 {Binomial coefficients}

\begin{exercise}
  An array of binomial coefficients can be computed as follows:
  %
  \cverbatimsnippet[code/omp/c/binomial1.c]{binomialarray}
  %
  Putting a single task group around the double loop, and use
  \n{depend} clauses to make the execution satisfy the proper dependencies.
\end{exercise}

\index{OpenMP!tasks|)}

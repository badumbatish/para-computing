% -*- latex -*-
%%%%%%%%%%%%%%%%%%%%%%%%%%%%%%%%%%%%%%%%%%%%%%%%%%%%%%%%%%%%%%%%
%%%%%%%%%%%%%%%%%%%%%%%%%%%%%%%%%%%%%%%%%%%%%%%%%%%%%%%%%%%%%%%%
%%%%
%%%% This text file is part of the source of 
%%%% `Parallel Programming in MPI and OpenMP'
%%%% by Victor Eijkhout, copyright 2012-9
%%%%
%%%% mpi-bcastreduce.tex : about broadcast & reduce collectives
%%%%
%%%%%%%%%%%%%%%%%%%%%%%%%%%%%%%%%%%%%%%%%%%%%%%%%%%%%%%%%%%%%%%%
%%%%%%%%%%%%%%%%%%%%%%%%%%%%%%%%%%%%%%%%%%%%%%%%%%%%%%%%%%%%%%%%

\Level 0 {Variable-size-input collectives}
\label{sec:v-collective}

In the gather and scatter call above each processor received or sent
an identical number of items. In many cases this is appropriate, but
sometimes each processor wants or contributes an individual number of
items. 

Let's take the gather calls as an example. Assume that each processor 
does a local computation that produces a number of data elements,
and this number is different for each processor (or at least not
the same for all). In the regular \indexmpishow{MPI_Gather} call the root processor
had a buffer of size~$nP$, where $n$~is the number of elements produced
on each processor, and $P$~the number of processors. The contribution
from processor~$p$ would go into locations $pn,\ldots,(p+1)n-1$.

For the variable case, we first need to compute the total required
buffer size. This can be done through a simple \indexmpishow{MPI_Reduce}
with \indexmpishow{MPI_SUM} as reduction operator:
the buffer size is $\sum_p n_p$ where $n_p$~is the number of elements
on processor~$p$. But you can also postpone
this calculation for a minute. 

The next question is where the contributions of the processor will
go into this buffer. For the contribution from processor~$p$
that is $\sum_{q<p}n_p,\ldots\sum_{q\leq p}n_p-1$. To compute this,
the root processor needs to have all the $n_p$ numbers, and it can collect
them with an \indexmpishow{MPI_Gather} call.

We now have all the ingredients.
All the processors specify a send buffer just as with \indexmpishow{MPI_Gather}.
However, the receive buffer specification on the root is more complicated. 
It now consists of:
\begin{verbatim}
outbuffer, array-of-outcounts, array-of-displacements, outtype
\end{verbatim}
and you have just seen how to construct that information.

For example, in an \indexmpiref{MPI_Gatherv} call each process has an individual
number of items to contribute. To gather this, the root process needs
to find these individual amounts with an \indexmpishow{MPI_Gather} call, and
locally construct the offsets array. Note how the offsets array has
size \n{ntids+1}: the final offset value is automatically the total
size of all incoming data. See the example below.

There are various calls where processors can have
buffers of differing sizes.
\begin{itemize}
\item In \indexmpishow{MPI_Scatterv} the root process has a different
  amount of data for each recipient.
\item In \indexmpishow{MPI_Gatherv}, conversely, each process
  contributes a different sized send buffer to the received result;
  \indexmpishow{MPI_Allgatherv} does the same, but leaves its result
  on all processes; \indexmpishow{MPI_Alltoallv} does a different
  variable-sized gather on each process.
\end{itemize}

\begin{lstlisting}
int MPI_Scatterv
  (void* sendbuf, int *sendcounts, int *displs, MPI_Datatype sendtype, 
   void* recvbuf, int recvcount, MPI_Datatype recvtype, 
   int root, MPI_Comm comm)
\end{lstlisting}

\begin{lstlisting}
int MPI_Allgatherv
  (void *sendbuf, int sendcount, MPI_Datatype sendtype, 
   void *recvbuf, int *recvcounts, int *displs, 
   MPI_Datatype recvtype, MPI_Comm comm)
\end{lstlisting}

\Level 1 {Example of Gatherv}

We use \indexmpishow{MPI_Gatherv} to do an irregular gather onto a root. We first need an
\indexmpishow{MPI_Gather} to determine offsets.
%
\cverbatimsnippet[examples/mpi/c/gatherv.c]{gatherv}
%
\pverbatimsnippet[examples/mpi/p/gatherv.py]{gathervp}

\Level 1 {Example of Allgatherv}

Prior to the actual gatherv call, we need to construct the count and
displacement arrays. The easiest way is to use a reduction.

\cverbatimsnippet[examples/mpi/c/allgatherv.c]{allgathervc}

In python the receive buffer has to contain the counts and
displacements arrays.

\pverbatimsnippet[examples/mpi/p/allgatherv.py]{bufallocp}

\pverbatimsnippet[examples/mpi/p/allgatherv.py]{allgathervp}

\Level 1 {Variable all-to-all}

\indexmpiref{MPI_Alltoallv}


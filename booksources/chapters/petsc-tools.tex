% -*- latex -*-
%%%%%%%%%%%%%%%%%%%%%%%%%%%%%%%%%%%%%%%%%%%%%%%%%%%%%%%%%%%%%%%%
%%%%%%%%%%%%%%%%%%%%%%%%%%%%%%%%%%%%%%%%%%%%%%%%%%%%%%%%%%%%%%%%
%%%%
%%%% This text file is part of the source of 
%%%% `Parallel Programming in MPI and OpenMP'
%%%% by Victor Eijkhout, copyright 2012-2021
%%%%
%%%% petsc-tools.tex : guess.
%%%%
%%%%%%%%%%%%%%%%%%%%%%%%%%%%%%%%%%%%%%%%%%%%%%%%%%%%%%%%%%%%%%%%
%%%%%%%%%%%%%%%%%%%%%%%%%%%%%%%%%%%%%%%%%%%%%%%%%%%%%%%%%%%%%%%%

\Level 0 {Error checking and debugging}

\Level 1 {Debug mode}
\label{sec:petsc-debug-mode}

During installation (see section~\ref{sec:petsc-install}),
there is an option of turning on debug mode.
An installation with debug turned on:
\begin{itemize}
\item Does more runtime checks on numerics, or array indices;
\item Does a memory analysis when you insert the \indexpetscshow{CHKMEMQ} macro
  (section~\ref{sec:petsc-mem});
\item Has the macro \indexpetscdef{PETSC_USE_DEBUG} set to~1.
\end{itemize}

\Level 1 {Error codes}
\label{sec:petsc-error}

PETSc performs a good amount of runtime error checking. Some of this
is for internal consistency, but it can also detect certain
mathematical errors. To facilitate error reporting, the following
scheme is used.
\begin{enumerate}
\item Every PETSc routine is a function returning a parameter of type
  \indexpetscdef{PetscErrorCode}.
\item For a good traceback, surround the executable part of any subprogram
  with \indexpetscdef{PetscFunctionBegin} and \indexpetscdef{PetscFunctionReturn},
  where the latter has the return value as parameter.
\item Calling the macro \indexpetscdef{CHKERRQ} on the error code will
  cause an error to be printed and the current routine to be
  terminated. Recursively this gives a traceback of where the error
  occurred.
\begin{lstlisting}
PetscErrorCode ierr;
ierr = AnyPetscRoutine( arguments ); CHKERRQ(ierr);
\end{lstlisting}
\item Other error checking macros are \indexpetscdef{CHKERRABORT}
  which aborts immediately, and \indexpetscdef{CHKERRMPI}.
\item You can effect your own error return by using
  \indexpetscref{SETERRQ}
  \indexpetscxref{SETERRQ1}{SETERRQ},
  \indexpetscxref{SETERRQ2}{SETERRQ}.
\end{enumerate}

\begin{fortrannote}
  In the main program, use \indexpetscdef{CHKERRA} and
  \indexpetscdef{SETERRA}. Also beware that these error `commands' are
  macros, and after expansion may interfere with
  \indextermbus{Fortran}{line length}, so they should only be
  used in \n{.F90} files.
\end{fortrannote}

Example. We write a routine that sets an error:
\cverbatimsnippet[examples/petsc/c/backtrace.c]{petscbacktrace}

Running this gives, in process~zero, the output
\begin{verbatim}
[0]PETSC ERROR: We cannot go on like this
[0]PETSC ERROR: See https://www.mcs.anl.gov/petsc/documentation/faq.html for trouble shooting.
[0]PETSC ERROR: Petsc Release Version 3.12.2, Nov, 22, 2019
[0]PETSC ERROR: backtrace on a [computer name]
[0]PETSC ERROR: Configure options [all options]
[0]PETSC ERROR: #1 this_function_bombs() line 20 in backtrace.c
[0]PETSC ERROR: #2 main() line 30 in backtrace.c  
\end{verbatim}

\begin{fortrannote}
  In Fortran the backtrace is not quite as elegant.
  \fverbatimsnippet[examples/petsc/f/backtrace.F90]{petscbacktracef}
\begin{verbatim}
[0]PETSC ERROR: ----- Error Message ------------------------------
[0]PETSC ERROR: We cannot go on like this
[....]
[0]PETSC ERROR: #1 User provided function() line 0 in User file
\end{verbatim}
\end{fortrannote}

\begin{remark}
  In this example, the use of \indexpetscshow{PETSC_COMM_SELF} indicates
  that this error is individually generated on a process;
  use \indexpetscshow{PETSC_COMM_WORLD} only if the same error would
  be detected everywhere.
\end{remark}

\begin{exercise}
  Look up the definition of \indexpetscshow{SETERRQ1}.
  Write a routine to compute square roots that is used as follows:
\begin{lstlisting}
  x = 1.5; ierr = square_root(x,&rootx); CHKERRQ(ierr);
  PetscPrintf(PETSC_COMM_WORLD,"Root of %f is %f\n",x,rootx);
  x = -2.6; ierr = square_root(x,&rootx); CHKERRQ(ierr);
  PetscPrintf(PETSC_COMM_WORLD,"Root of %f is %f\n",x,rootx);
\end{lstlisting}
This should give as output:
\begin{verbatim}
Root of 1.500000 is 1.224745
[0]PETSC ERROR: ----- Error Message ----------------------------------------------
[0]PETSC ERROR: Cannot compute the root of -2.600000
[...]
[0]PETSC ERROR: #1 square_root() line 23 in root.c
[0]PETSC ERROR: #2 main() line 39 in root.c
\end{verbatim}

\end{exercise}

\Level 1 {Memory corruption}
\label{sec:petsc-mem}

PETSc has its own memory management (section~\ref{sec:petscmalloc})
and this facilitates finding memory corruption errors.
The macro \indexpetscdef{CHKMEMQ} (\indexpetscdef{CHKMEMA} in void functions)
checks all memory that was allocated by PETSc,
either internally or throug the allocation routines,
for corruption. Sprinkling this macro through your code
can detect memory problems before they lead to a \indexterm{segfault}.

This testing is only done if the commandline argument
\indextermtt{-malloc_debug} (\indextermtt{-malloc_test} in debug mode)
is supplied, so it carries no overhead for production runs.

\Level 2 {Valgrind}

Valgrind is rather verbose in its output.
To limit the number of processs that run under valgrind:
\begin{verbatim}
mpiexec -n 3 valgrind --track-origins=yes ./app -args : -n 5 ./app -args
\end{verbatim}

\Level 0 {Program output}

PETSc has as variety of mechanisms to export or visualize program data.
We will consider a few possibilities here.

\Level 1 {Screen I/O}

Printing screen output in parallel is tricky. If two processes execute
a print statement at more or less the same time there is no guarantee
as to in what order they may appear on screen. (Even attempts to have
them print one after the other may not result in the right ordering.)
Furthermore, lines from multi-line print actions on two processes may
wind up on the screen interleaved.

\Level 2 {printf replacements}

PETSc has two routines that fix this problem. First of all, often the
information printed is the same on all processes, so it is enough if
only one process, for instance process~0, prints it. This is done with
\indexpetscref{PetscPrintf}.

If all processes need to print, you can use
\indexpetscref{PetscSynchronizedPrintf} that forces the
output to appear in process order.

To make sure that output is properly flushed from all system buffers
use \indexpetscref{PetscSynchronizedFlush}
where for ordinary screen output you would use \n{stdout} for the file.

\begin{fortrannote}
  The Fortran calls are only wrappers around C~routines, so you can 
  use \verb+\n+ newline characters in the Fortran string argument
  to \indexpetscshow{PetscPrintf}.

  The file to flush is typically \indexpetscshow{PETSC_STDOUT}.
\end{fortrannote}

\begin{pythonnote}{Petsc print and python print}
  Since the print routines use the python \n{print} call, they
  automatically include the trailing newline. You don't have to
  specify it as in the C~calls.
\end{pythonnote}

\Level 2 {scanf replacement}

Using \indextermtt{scanf} in Petsc is tricky, since integers and real numbers can be
of different sizes, depending on the installation.
Instead, use \indexpetscref{PetscViewerRead}, which operates in terms of
\indexpetscshow{PetscDataType}.

\Level 1 {Exporting internal data structures}
\label{sec:petsc-view}

In order to export PETSc matrix or vector data structures
there is a \indexpetscdef{PetscViewer} object type.
This is a quite general concept of viewing: it encompasses ascii output to screen,
binary dump to file, or communication to a running Matlab process.
Calls such as  \indexpetscshow{MatView} or \indexpetscshow{KSPView}
accept a \lstinline{PetscViewer} argument.

In cases where this makes sense, there is also an inverse `load' operation.
See section \ref{sec:vecviewload} for vectors.

Some viewers are predefined, such as
\indexpetscdef{PETSC_VIEWER_STDOUT_WORLD} for ascii rendering to
standard out. (In~C, specifying zero or \n{NULL} also uses this
default viewer; for Fortran use \indexpetsctt{PETSC_NULL_VIEWER}.)

\Level 2 {Viewer types}

For activities such as dumping to file
you first need create the viewer
with \indexpetscdef{PetscViewerCreate}
and set its type with \indexpetscdef{PetscViewerSetType}.
\begin{lstlisting}
PetscViewerCreate(comm,&viewer);
PetscViewerSetType(viewer,PETSCVIEWERBINARY);  
\end{lstlisting}
Popular types include
\indexpetscdef{PETSCVIEWERASCII},
\indexpetscdef{PETSCVIEWERBINARY},
\indexpetscdef{PETSCVIEWERSTRING},
\indexpetscdef{PETSCVIEWERDRAW},
\indexpetscdef{PETSCVIEWERSOCKET},
\indexpetscdef{PETSCVIEWERHDF5},
\indexpetscdef{PETSCVIEWERVTK};
the full list can be found in \n{include/petscviewer.h}.

\Level 2 {Viewer formats}

Viewers can take further format specifications by using
\indexpetscdef{PetscViewerPushFormat}:
\begin{lstlisting}
PetscViewerPushFormat
   (PETSC_VIEWER_STDOUT_WORLD,
    PETSC_VIEWER_ASCII_INFO_DETAIL);
\end{lstlisting}
and afterwards a corresponding \indexpetscdef{PetscViewerPopFormat}

\Level 2 {Commandline option for viewers}

Petsc objects viewers can be activated by calls such as \indexpetscshow{MatView},
but often it is more convenient to do this through commandline options,
such as \indexpetscoption{mat_view}, \indexpetscoption{vec_view},
or \indexpetscoption{ksp_view}.
By default, these output to \n{stdout} in \n{ascii} form,
but this can be controlled by further option values:
\begin{verbatim}
program -mat_view binary:matrix.dat
\end{verbatim}
where \n{binary} forces a binary dump (\n{ascii} is the default)
and a file name is explicitly given.

If a viewer needs to be triggered at a specific location,
calls such as \indexpetscdef{VecViewFromOptions} can be used.
%% NO probably in combination with \indexpetscshow{VecSetOptionsPrefix}.
These routines all have a similar calling sequence:
\begin{lstlisting}
#include "petscsys.h"    
PetscErrorCode  PetscObjectViewFromOptions(PetscObject obj,PetscObject bobj,const char optionname[])
PetscErrorCode  VecViewFromOptions(Vec A,PetscObject obj,const char name[])
\end{lstlisting}

\begin{raggedlist}
  \indexpetscshow{AOViewFromOptions},
  \indexpetscshow{DMViewFromOptions},
  \indexpetscshow{ISViewFromOptions},
  \indexpetscshow{ISLocalToGlobalMappingViewFromOptions},
  \indexpetscshow{KSPConvergedReasonViewFromOptions},
  \indexpetscshow{KSPViewFromOptions},
  \indexpetscshow{MatPartitioningViewFromOptions},
  \indexpetscshow{MatCoarsenViewFromOptions},
  \indexpetscshow{MatViewFromOptions},
  \indexpetscshow{PetscObjectViewFromOptions},
  \indexpetscshow{PetscPartitionerViewFromOptions},
  \indexpetscshow{PetscDrawViewFromOptions},
  \indexpetscshow{PetscRandomViewFromOptions},
  \indexpetscshow{PetscDualSpaceViewFromOptions},
  \indexpetscshow{PetscSFViewFromOptions},
  \indexpetscshow{PetscFEViewFromOptions},
  \indexpetscshow{PetscFVViewFromOptions},
  \indexpetscshow{PetscSectionViewFromOptions},
  \indexpetscshow{PCViewFromOptions},
  \indexpetscshow{PetscSpaceViewFromOptions},
  \indexpetscshow{PFViewFromOptions},
  \indexpetscshow{PetscLimiterViewFromOptions},
  \indexpetscshow{PetscLogViewFromOptions},
  \indexpetscshow{PetscDSViewFromOptions},
  \indexpetscshow{PetscViewerViewFromOptions},
  \indexpetscshow{SNESConvergedReasonViewFromOptions},
  \indexpetscshow{SNESViewFromOptions},
  \indexpetscshow{TSTrajectoryViewFromOptions},
  \indexpetscshow{TSViewFromOptions},
  \indexpetscshow{TaoLineSearchViewFromOptions},
  \indexpetscshow{TaoViewFromOptions},
  \indexpetscshow{VecViewFromOptions},
  \indexpetscshow{VecScatterViewFromOptions},
\end{raggedlist}

\Level 2 {Naming objects}

A helpful facility for viewing is to name an object:
that name will then be displayed when the object is viewed.
\begin{lstlisting}
Vec i_local;
ierr = VecCreate(comm,&i_local); CHKERRQ(ierr);
ierr = PetscObjectSetName((PetscObject)i_local,"space local"); CHKERRQ(ierr);
\end{lstlisting}
giving:
\begin{verbatim}
Vec Object: space local 4 MPI processes
  type: mpi
Process [0]
[ ... et cetera ... ]
\end{verbatim}

\Level 0 {Commandline options}
\label{sec:petsc-options}

PETSc has as large number of commandline options, most of which we
will discuss later. For now we only mention \indexpetscshow{-log_summary} which
will print out profile of the time taken in various routines.
For these options to be parsed, it is necessary to pass \n{argc,argv}
to the \indexpetscshow{PetscInitialize} call.

\Level 1 {Adding your own options}

You can add custom 
commandline options to your program.
Various routines such as \indexpetscdef{PetscOptionsGetInt}
scan the commandline for options and set parameters accordingly.
For instance,
%
\cverbatimsnippet[examples/petsc/c/kspcg.c]{optionint}
%
declares the existence of an option \n{-n} to be followed by an integer.

Now executing
\begin{verbatim}
mpiexec yourprogram -n 5
\end{verbatim}
will
\begin{enumerate}
\item set the \lstinline{flag} to true, and
\item set the parameter \lstinline{domain_size} to the value on the commandline.
\end{enumerate}
Omitting the \n{-n} option will leave the default value of \lstinline{domain_size}
unaltered.

For flags, use \indexpetscdef{PetscOptionsHasName}.

\begin{pythonnote}{Petsc options}
  In Python, do not specify the initial hyphen of an option name.
  Also, the functions such as \lstinline{getInt} do not return the boolean flag;
  if you need to test for the existence of the commandline option, use:
\begin{verbatim}
hasn = PETSc.Options().hasName("n")
\end{verbatim}
\end{pythonnote}

There is a related mechanism using
\indexpetscdef{PetscOptionsBegin}~/ \indexpetscdef{PetscOptionsEnd}:
\cverbatimsnippet[examples/petsc/c/optionsbegin.c]{petscoptionsblock}

The selling point for this approach is that running your code with
\begin{verbatim}
mpiexec yourprogram -help
\end{verbatim}
will display these options as a block.
Together with a ton of other options, unfortunately.

\Level 1 {Options prefix}

In many cases, your code will have only one \lstinline{KSP} solver object,
so specifying \indexpetscoption{ksp_view} or \indexpetscoption{ksp_monitor}
will display~/ trace that one.
However, you may have multiple solvers, or nested solvers. You may then
not want to display all of them.

As an example of the nest solver case, consider the case of a
\indextermsub{block jacobi}{preconditioner}, where the block is itself solved
with an iterative method. You can trace that one with
\indexpetscoption{-sub_ksp_monitor}.

The \n{sub_} is an \indextermbus{option}{prefix}, and you can defined
your own with 
\indexpetscdef{KSPSetOptionsPrefix}.
(There are similar routines for other PETSc object types.)

Example:
\begin{lstlisting}
KSPCreate(comm,&time_solver);
KSPCreate(comm,&space_solver);
KSPSetOptionsPrefix(time_solver,"time_");
KSPSetOptionsPrefix(space_solver,"space_");
\end{lstlisting}
You can then use options \n{-time_ksp_monitor} and such.
Note that the prefix does not have a leading dash,
but it does have the trailing underscore.

\begin{raggedlist}
Similar routines:
  \indexpetscdef{MatSetOptionsPrefix}, 
  \indexpetscdef{PCSetOptionsPrefix}, 
  \indexpetscdef{PetscObjectSetOptionsPrefix}, 
  \indexpetscdef{PetscViewerSetOptionsPrefix}, 
  \indexpetscdef{SNESSetOptionsPrefix}, 
  \indexpetscdef{TSSetOptionsPrefix}, 
  \indexpetscdef{VecSetOptionsPrefix}, 
and some more obscure ones.
\end{raggedlist}


\Level 1 {Where to specify options}
\label{sec:petscrc}

Commandline options can obviously go on the commandline. However, there are more
places where they can be specified.

Options can be specified programmatically with
\indexpetscdef{PetscOptionsSetValue}:
\begin{lstlisting}
PetscOptionsSetValue( NULL, // for global options
  "-some_option","value_as_string");
\end{lstlisting}

Options can be specified in a file \indextermtt{.petscrc} in the user's home directory
or the current directory.

Finally, an environment variable \indextermtt{PETSC_OPTIONS} can be set.

The \n{rc} file is processed first, then the environment variable,
then any commandline arguments. This parsing is done in \indexpetscshow{PetscInitialize},
so any values from \indexpetscshow{PetscOptionsSetValue} override this.

\Level 0 {Timing and profiling}
\label{sec:petsc-timing}

PETSc has a number of timing routines that make it unnecessary to
use system routines such as \indextermtt{getrusage}
or MPI routines such as \indexmpishow{MPI_Wtime}.
The main (wall clock) timer is \indexpetscref{PetscTime}.
Note the return type of \indexpetscdef{PetscLogDouble} which
can have a different precision from \indexpetscshow{PetscReal}.

The routine \indexpetscdef{PetscGetCPUTime} is less useful, since it measures only
time spent in computation, and ignores things such as communication.

\Level 1 {Logging}

Petsc does a lot of logging on its own operations.
Additionally, you can introduce your own routines into this log.

The simplest way to display statistics is to run
with an option \indexpetscoption{log_view}.
This takes an optional file name argument:
\begin{verbatim}
mpiexec -n 10 yourprogram -log_view :statistics.txt
\end{verbatim}
The corresponding routine is \indexpetscdef{PetscLogView}.

\Level 0 {Memory management}
\label{sec:petscmalloc}

Allocate the memory for a given pointer: \indexpetscdef{PetscNew},
allocate arbitrary memory with \indexpetscdef{PetscMalloc},
allocate a number of objects with \indexpetscref{PetscMalloc1}
(this does not zero the memory allocated,
 use \indexpetscdef{PetscCalloc1} to obtain memory that has been zeroed);
use \indexpetscref{PetscFree} to free.
\begin{lstlisting}
PetscInt *idxs;
PetscMalloc1(10,&idxs);
// better than:
// PetscMalloc(10*sizeof(PetscInt),&idxs);
for (PetscInt i=0; i<10; i++)
  idxs[i] = f(i);
PetscFree(idxs);
\end{lstlisting}
Allocated memory is aligned to \indexpetscdef{PETSC_MEMALIGN}.

The state of memory allocation can be written to file or standard out
with \indexpetscdef{PetscMallocDump}. The commandline option
\indexpetscshow{-malloc_dump} outputs all not-freed memory during
\indexpetscshow{PetscFinalize}.

\Level 1 {GPU allocation}

The memories of a CPU and GPU are not coherent.
This means that routines such as \indexpetscshow{PetscMalloc1}
can not immediately be used for GPU allocation.
See section~\ref{sec:petsc-malloc-gpu} for details.

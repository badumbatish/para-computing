% -*- latex -*-
%%%%%%%%%%%%%%%%%%%%%%%%%%%%%%%%%%%%%%%%%%%%%%%%%%%%%%%%%%%%%%%%
%%%%%%%%%%%%%%%%%%%%%%%%%%%%%%%%%%%%%%%%%%%%%%%%%%%%%%%%%%%%%%%%
%%%%
%%%% This text file is part of the source of 
%%%% `Parallel Programming in MPI and OpenMP'
%%%% by Victor Eijkhout, copyright 2012-2020
%%%%
%%%% petsc.tex : petsc stuff
%%%%
%%%%%%%%%%%%%%%%%%%%%%%%%%%%%%%%%%%%%%%%%%%%%%%%%%%%%%%%%%%%%%%%
%%%%%%%%%%%%%%%%%%%%%%%%%%%%%%%%%%%%%%%%%%%%%%%%%%%%%%%%%%%%%%%%

\Level 0 {Communicators}

PETSc has a `world' communicator, which by default equals
\lstinline{MPI_COMM_WORLD}. If you want to run PETSc on a subset of processes,
you can assign a subcommunicator to the variable \lstinline{PETSC_COMM_WORLD}
in between the calls to \indexmpishow{MPI_Init} and
\indexpetscshow{PetscInitialize}.
Petsc communicators are of type \indexpetscdef{PetscComm}.

\Level 0 {Scalars}

The definition of
\indexpetscdef{PetscInt}, \indexpetscdef{PetscReal}, \indexpetscdef{PetscComplex}
depends on how PETSc was installed.
This makes interoperability with other libraries such as
\emph{MPI}%
\index{Petsc!interoperability with MPI}
a little tricky.

The equivalent \indexpetscshow{MPI_Datatype} values are
\indexpetscdef{PetscMPIInt}
\indexpetscdef{MPIU_INT}
\indexpetscdef{MPIU_REAL}
\indexpetscdef{MPIU_SCALAR}
\indexpetscdef{MPIU_COMPLEX}

Similarly, for the
\emph{BLAS}%
\index{Petsc!interoperability with BLAS}
library there  are:
%
\indexpetscdef{PetscBLASInt}

No real types are needed since these are passed by reference.

Furthermore, there is
\begin{lstlisting}
#define PETSC_BINARY_INT_SIZE    (32/8)
#define PETSC_BINARY_FLOAT_SIZE  (32/8)
#define PETSC_BINARY_CHAR_SIZE   (8/8)
#define PETSC_BINARY_SHORT_SIZE  (16/8)
#define PETSC_BINARY_DOUBLE_SIZE (64/8)
#define PETSC_BINARY_SCALAR_SIZE sizeof(PetscScalar)  
\end{lstlisting}

\endinput


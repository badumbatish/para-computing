% -*- latex -*-
%%%%%%%%%%%%%%%%%%%%%%%%%%%%%%%%%%%%%%%%%%%%%%%%%%%%%%%%%%%%%%%%
%%%%%%%%%%%%%%%%%%%%%%%%%%%%%%%%%%%%%%%%%%%%%%%%%%%%%%%%%%%%%%%%
%%%%
%%%% This text file is part of the source of 
%%%% `Parallel Programming in MPI and OpenMP'
%%%% by Victor Eijkhout, copyright 2012-2020
%%%%
%%%% mpi-commbasic.tex : communicator basics
%%%%
%%%%%%%%%%%%%%%%%%%%%%%%%%%%%%%%%%%%%%%%%%%%%%%%%%%%%%%%%%%%%%%%
%%%%%%%%%%%%%%%%%%%%%%%%%%%%%%%%%%%%%%%%%%%%%%%%%%%%%%%%%%%%%%%%

With \indexmpishow{MPI_Comm_dup} you can make an exact duplicate of a communicator.
This may seem pointless, but it is actually very useful for the design of
software libraries. Image that you have a code
\lstset{style=reviewcode,language=C}
\begin{lstlisting}
MPI_Isend(...); MPI_Irecv(...);
// library call
MPI_Waitall(...);
\end{lstlisting}
and suppose that the library has receive calls. Now it is possible that the 
receive in the library inadvertently
catches the message that was sent in the outer environment.

In section~\ref{sec:mpi-semantics} it was explained that MPI messages are 
non-overtaking. This may lead to confusing situations, witness the following.
First of all, here is code where the library stores the communicator
of the calling program:
%
\cxxverbatimsnippet[examples/mpi/c/commdupwrong.cxx]{wrongcatchlib}

This models a main program that does a simple message exchange, and it
makes two calls to library routines. Unbeknown to the user, the
library also issues send and receive calls, and they turn out to
interfere.

Here
\begin{itemize}
\item The main program does a send,
\item the library call \n{function_start} does a send and a receive;
  because the receive can match either send, it is paired with the
  first one;
\item the main program does a receive, which will be paired with the send of the 
  library call;
\item both the main program and the library do a wait call, and in
  both cases all requests are succesfully fulfilled, just not the way
  you intended.
\end{itemize}

To prevent this confusion, the library should duplicate the outer
communicator with
%
\indexmpiref{MPI_Comm_dup}
%
and send all messages with respect to its duplicate. Now messages from the user
code can never reach the library software, since they are on different communicators.

\cverbatimsnippet[examples/mpi/c/commdupright.cxx]{rightcatchlib}

Note how the preceding example
performs the \indexmpishow{MPI_Comm_free}
cal in a C++ \indexterm{destructor}.

\pverbatimsnippet[examples/mpi/p/commdup.py]{catchlibp}


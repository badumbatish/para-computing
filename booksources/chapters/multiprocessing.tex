% -*- latex -*-
%%%%%%%%%%%%%%%%%%%%%%%%%%%%%%%%%%%%%%%%%%%%%%%%%%%%%%%%%%%%%%%%
%%%%%%%%%%%%%%%%%%%%%%%%%%%%%%%%%%%%%%%%%%%%%%%%%%%%%%%%%%%%%%%%
%%%%
%%%% This text file is part of the source of 
%%%% `Parallel Programming in MPI and OpenMP'
%%%% by Victor Eijkhout, copyright 2012-2022
%%%%
%%%% multiprocessing.tex : python multiprocessing module
%%%%
%%%%%%%%%%%%%%%%%%%%%%%%%%%%%%%%%%%%%%%%%%%%%%%%%%%%%%%%%%%%%%%%
%%%%%%%%%%%%%%%%%%%%%%%%%%%%%%%%%%%%%%%%%%%%%%%%%%%%%%%%%%%%%%%%

\lstset{language=Python}

Python has a \indexterm{multiprocessing} toolbox.
This is a parallel processing library that relies on subprocesses,
rather than threads.

\Level 0 {Software and hardware}

The \n{multiprocessing} toolbox does its own hardware detection;
it uses a single node, and however many cores the system tells it
there are.

\pverbatimsnippet{pycpycount}

\Level 0 {Process}

A process is an object that will execute a python function:
\pverbatimsnippet[code/multiprocessing/p/hello.py]{pyproccreate}

Creating a process does not start it:
for that use the \lstinline{start} function.
Execution of the process is not guaranteed until you call the \lstinline{join}
function on it:
\pverbatimsnippet[code/multiprocessing/p/hello.py]{pyprocstartjoin}

By making the start and join calls less regular than
in a loop like this,
arbitrarily complicated code can be produced.

\Level 1 {Arguments}

Arguments can be passed to the function of the process
with the \lstinline{args} keyword.
This accepts a list (or tuple) of arguments,
leading to a somewhat strange syntax for a single argument:
\begin{lstlisting}
proc = Process(target=print_func, args=(name,))
\end{lstlisting}

\Level 1 {Process details}

Note the test on \n{__main__}:
the processes started read the current file in order to execute
the function specified.
Without this clause, the import would first execute more process start calls,
before getting to the function execution.

Processes have a name that you can retrieve as
\lstinline+current_process().name+.
The default is \n{Process-5} and such,
but you can specify custom names:
\begin{lstlisting}
Process(name="Your name here")
\end{lstlisting}
The target function of a process can get hold of that process
with the  \lstinline+current_process+ function.

Of course you can also query \lstinline+os.getpid()+
but that does not offer any further possibilities.

\pverbatimsnippet[code/multiprocessing/p/hello.py]{pyprocpid}

\Level 0 {Pools and mapping}

Often you want a number of processes to do apply to a number of arguments,
for instance in a \indexterm{parameter sweep}.
For this, create a \lstinline{Pool} object,
and apply the \lstinline{map} method to it:
%
\pverbatimsnippet[code/multiprocessing/p/pool.py]{pymulpoolmap}

Note that this is also the easiest way to get return values from a process,
which is not directly possible with a \lstinline{Process} object.
Other approaches are using a shared object,
or an object in a \lstinline{Queue} or \lstinline{Pipe} object; see below.

\Level 0 {Shared data}

The best way to deal with shared data is to create
a \lstinline{Value} or \lstinline{Array} object,
which come equipped with a lock for safe updating.

\pverbatimsnippet{pypivalue}

For instance, one could stochastically calculate~$\pi$
by 
\begin{enumerate}
\item generating random points in~$[0,1)^2$, and
\item recording how many fall in the unit circle, after which
\item $\pi$ is $4\times$ the ratio between points in the circle
  and the total number of points.
\end{enumerate}

\pverbatimsnippet{pypicompute}

\begin{exercise}
  Do you see a way to improve the speed of this calculation?
\end{exercise}

\Level 1 {Pipes}

A \indexterm{pipe}, object type \lstinline{Pipe},
corresponds to what used to be called a \indexterm{channel}
in older parallel programming systems:
a~\ac{FIFO} object into which one process can place items,
and from which another process can take them.
However, a pipe is not associated with any particular pair:
creating the pipe gives the entrace and exit from the pipe

\pverbatimsnippet{pypipecreate}

And they can be passed to any process

\pverbatimsnippet{pypipepass}

which can then can put and get items,
using the \lstinline{send} and \lstinline{recv} commands.

\pverbatimsnippet{pypipesend}
\pverbatimsnippet{pypiperecv}

\Level 1 {Queues}

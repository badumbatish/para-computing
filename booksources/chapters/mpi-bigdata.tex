% -*- latex -*-
%%%%%%%%%%%%%%%%%%%%%%%%%%%%%%%%%%%%%%%%%%%%%%%%%%%%%%%%%%%%%%%%
%%%%%%%%%%%%%%%%%%%%%%%%%%%%%%%%%%%%%%%%%%%%%%%%%%%%%%%%%%%%%%%%
%%%%
%%%% This text file is part of the source of 
%%%% `Parallel Programming in MPI and OpenMP'
%%%% by Victor Eijkhout, copyright 2012-2022
%%%%
%%%% mpi-bigdata.tex : discussion of MPI big datatypes
%%%%
%%%%%%%%%%%%%%%%%%%%%%%%%%%%%%%%%%%%%%%%%%%%%%%%%%%%%%%%%%%%%%%%
%%%%%%%%%%%%%%%%%%%%%%%%%%%%%%%%%%%%%%%%%%%%%%%%%%%%%%%%%%%%%%%%

\Level 0 {Big data types}
\label{sec:mpi-bigdata}
\index{datatype!big|(}

The \n{size} parameter in MPI send and receive calls is of type integer,
meaning that it's maximally
(platform-dependent, but typically:)
$2^{31}-1$.
These day computers are big enough
that this is a limitation.
As of the \mpistandard{4} standard, this has been solved by allowing a
larger count parameter.
The implementation of this depends somewhat on the language.

\begin{mpifournote}{MPI Count type}

\Level 1 {C}
\label{sec:c-mpi-count}

For every routine, such as \indexmpishow{MPI_Send} with an integer count,
there is a corresponding \indexmpishow{MPI_Send_c} with a count
of type \indexmpishow{MPI_Count}.

\cverbatimsnippet{reducecount}

{
\def\snippetcodefraction{.45}
\def\snippetlistfraction{.5}  
\csnippetwithoutput{bigpingpong}{examples/mpi/c}{pingpongbig}
}

\Level 1 {Fortran}
\label{sec:f-largecount}

The count parameter can be declared  to be
\lstset{language=Fortran}
\begin{lstlisting}
use mpi_f08
Integer(kind=MPI_COUNT_KIND) :: count
\end{lstlisting}
\lstset{language=C}
Since Fortran has polymorphism, the same routine names can be used.

\begin{multicols}{2}
  The legit way of coding:
  %
  \lstset{xleftmargin=0pt}
  \fverbatimsnippet{ftypecheck}
  \columnbreak
  \ldots~but you can see what's under the hood:
  %
  \lstset{xleftmargin=0pt}
  \fverbatimsnippet{ftypecheck8}
\end{multicols}

Routines using this type are not available
unless using the \indextermtt{mpi_f08} module.
\end{mpifournote}

\fverbatimsnippet{bigdataf}

\Level 1 {Count datatype}

The \indexmpidef{MPI_Count} datatype is defined as being large enough
to accomodate values of
\begin{itemize}
\item the ordinary 4-byte integer type;
\item the \indexmpishow{MPI_Aint} type, sections
  \ref{sec:mpi-byte-type} and~\ref{sec:mpi-byte-type};
\item the \indexmpishow{MPI_Offset} type, section~\ref{sec:mpi-filepoint}.
\end{itemize}

The \lstinline+size_t+ type in~C/C++ is defined as big enough to
contain the output of \lstinline+sizeof+, that is,
being big enough to measure any object.

\Level 1 {MPI 3 temporary solution}

Large messages were already possible by using 
derived types: to send
a \emph{big data type} of $10^{40}$ elements you would
\begin{itemize}
\item create a contiguous type with $10^{20}$ elements, and
\item send $10^{20}$ elements of that type.
\end{itemize}
This often works, but it's not perfect. For instance, the routine
\indexmpishow{MPI_Get_elements} returns the total number of basic elements sent
(as opposed to \indexmpishow{MPI_Get_count} which would return the number
of elements of the derived type). Since its output argument is
of integer type, it can't store the right value.

The \mpistandard{3} standard has addressed this
through the introduction of an \indexmpishow{MPI_Count} datatype,
and new routines that return that type of count.
(In view of the `embiggened' routins, this solution
is no longer needed, and will probably be deprecated in later standards.)

Let us consider an example.

Allocating a buffer of more than 4Gbyte is not hard:
\cverbatimsnippet[examples/mpi/c/vectorx.c]{bigvectoralloc}

We use the trick with sending elements of a derived type:
\cverbatimsnippet[examples/mpi/c/vectorx.c]{bigvectorptp}

We use the same trick for the receive call, but now we catch the status
parameter which will later tell us how many elements of the basic type
were sent:
%
\cverbatimsnippet[examples/mpi/c/vectorx.c]{bigvectorrecv}

When we query how many of the basic elements are in the buffer
(remember that in the receive call the buffer length is
an upper bound on the number of elements received)
do we
need a counter that is larger than an integer.  MPI has introduced a
type \indexmpishow{MPI_Count} for this, and new routines such as
\indexmpixref{MPI_Get_elements_x}{MPI_Get_elements} that return a
count of this type:

\cverbatimsnippet[examples/mpi/c/vectorx.c]{bigvectorq}

\begin{remark}
  Computing a big number to allocate is not entirely simple.
  \cverbatimsnippet[examples/mpi/c/getx.c]{compsizet}
  gives as output:
\begin{verbatim}
size of size_t = 8
0 68719476736 68719476736 0 68719476736
\end{verbatim}
Clearly, not only do operations go left-to-right, but casting is done that way too:
the computed subexpressions are only cast to \lstinline{size_t} if one operand is.
\end{remark}

Above, we did not actually create a datatype that was bigger than~2G,
but if you do so, you can query its extent by
\indexmpixref{MPI_Type_get_extent_x}{MPI_Type_get_true_extent}
and
\indexmpixref{MPI_Type_get_true_extent_x}{MPI_Type_get_true_extent}.

\begin{pythonnote}{Big data}
  Since python has unlimited size integers there is
  no explicit need for the `x' variants of routines.
  Internally, \lstinline+MPI.Status.Get_elements+ is implemented
  in terms of \indexmpishow{MPI_Get_elements_x}.
\end{pythonnote}

\index{datatype!big|)}


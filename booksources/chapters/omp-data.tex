% -*- latex -*-
%%%%%%%%%%%%%%%%%%%%%%%%%%%%%%%%%%%%%%%%%%%%%%%%%%%%%%%%%%%%%%%%
%%%%%%%%%%%%%%%%%%%%%%%%%%%%%%%%%%%%%%%%%%%%%%%%%%%%%%%%%%%%%%%%
%%%%
%%%% This text file is part of the source of 
%%%% `Parallel Programming in MPI and OpenMP'
%%%% by Victor Eijkhout, copyright 2012-2022
%%%%
%%%% omp-data.tex : about thread data
%%%%
%%%%%%%%%%%%%%%%%%%%%%%%%%%%%%%%%%%%%%%%%%%%%%%%%%%%%%%%%%%%%%%%
%%%%%%%%%%%%%%%%%%%%%%%%%%%%%%%%%%%%%%%%%%%%%%%%%%%%%%%%%%%%%%%%

\label{sec:ompdata}

In a parallel region there are two types of data: private and shared.
In this sections we will see the various way you can control what category
your data falls under; for private data items we also discuss how their values
relate to shared data.

\Level 0 {Shared data}

In a parallel region, any data declared outside it will be shared:
any thread using a variable~\n{x} will access the same memory location
associated with that variable.

Example:
\begin{lstlisting}
  int x = 5;
#pragma omp parallel
  {
    x = x+1;
    printf("shared: x is %d\n",x);
  }
\end{lstlisting}
All threads increment the same variable, so after the loop it will
have a value of five plus the number of threads; or maybe less because of the data races
involved. See \HPSCref{sec:shared-lock} for an explanation of the issues involved;
see \ref{sec:critical} for a solution in OpenMP.

Sometimes this global update is what you want; in other cases the
variable is intended only for intermediate results in a computation.
In that case 
there are various ways of creating
data that is local to a thread, and therefore invisible to other threads.

\Level 0 {Private data}
\label{sec:omp-private}

In the C/C++ language it is possible to declare variables inside
a \indexterm{lexical scope}; roughly: inside curly braces.
This concept extends to OpenMP parallel regions and directives:
any variable declared in a block following an OpenMP directive
will be local to the executing thread.

Example:
\begin{lstlisting}
  int x = 5;
#pragma omp parallel
  {
    int x; x = 3;
    printf("local: x is %d\n",x);
  }
\end{lstlisting}
After the parallel region the outer variable~\n{x} will still have the
value~\n{5}: there is no \indexterm{storage association} between the
private variable and global one.

\begin{fortrannote}{Private variables in parallel region}
  The Fortran language does not have this concept of scope, so you have to use a
  \indexclause{private} clause:
\begin{lstlisting}
!$OMP parallel private(x)
\end{lstlisting}
\end{fortrannote}

The \indexpragma{private} directive declares data to have a separate copy 
in the memory of each thread. 
Such private variables are initialized as they would be in a main program.
Any computed value goes away at the end 
of the parallel region. (However, see below.)
Thus, you should not rely on any initial value, or on the value of the outer variable
after the region.

\begin{lstlisting}
  int x = 5;
#pragma omp parallel private(x)
  {
    x = x+1; // dangerous
    printf("private: x is %d\n",x);
  }
  printf("after: x is %d\n",x); // also dangerous
\end{lstlisting}

Data that is declared private with the \indexpragma{private} directive is
put on a separate \indextermbus{stack}{per thread}. The OpenMP standard
does not dictate the size of these stacks, but beware of \indextermbus{stack}{overflow}.
A~typical default
is a few megabyte; you can control it with the environment variable
\indexompshow{OMP_STACKSIZE}. Its values can be literal or with suffixes:
\begin{verbatim}
123 456k 567K 678m 789M 246g 357G
\end{verbatim}

A normal \indextermbus{Unix}{process} also has a stack, but this is
independent of the OpenMP stacks for private data. You can query or
set the Unix stack with \indextermtt{ulimit}:
\begin{verbatim}
[] ulimit -s
64000
[] ulimit -s 8192
[] ulimit -s
8192
\end{verbatim}
The Unix stack can grow dynamically as space is needed. This does not
hold for the OpenMP stacks: they are immediately allocated at their
requested size. Thus it is important not too make them too large.

\Level 0 {Data in dynamic scope}

Functions that are called from a parallel region fall in the
\emph{dynamic scope}\index{parallel region!dynamic scope} of that
parallel region. The rules for variables in that function are as follows:
\begin{itemize}
\item Any variables locally defined to the function are private.
\item \n{static} variables in C and \n{save} variables in Fortran
  are shared.
\item The function arguments inherit their status from the calling environment.
\end{itemize}

\Level 0 {Temporary variables in a loop}

It is common to have a variable that is set and used in each loop
iteration:
\begin{lstlisting}
#pragma omp parallel for
for ( ... i ... ) {
  x = i*h;
  s = sin(x); c = cos(x);
  a[i] = s+c;
  b[i] = s-c;
}
\end{lstlisting}
By the above rules, the variables \n{x,s,c} are all shared
variables. However, the values they receive in one iteration are not
used in a next iteration, so they behave in fact like private
variables to each iteration.
\begin{itemize}
\item In both C and Fortran you can declare these variables private in
  the parallel for directive.
\item In C, you can also redefine the variables inside the loop.
\end{itemize}

Sometimes, even if you forget to declare these temporaries as private,
the code may still give the correct output. That is because the
compiler can sometimes eliminate them from the loop body, since it detects that their
values are not otherwise used.

\Level 0 {Default}

\begin{itemize}
\item Loop variables in an \indexpragma{omp for} are private;
\item Local variables in the parallel region are private.
\end{itemize}
You can alter this default behavior with the \indexclause{default} clause:
\begin{lstlisting}
#pragma omp parallel default(shared) private(x)
{ ... }
#pragma omp parallel default(private) shared(matrix)
{ ... }
\end{lstlisting}
and if you want to play it safe:
\begin{lstlisting}
#pragma omp parallel default(none) private(x) shared(matrix)
{ ... }
\end{lstlisting}

\begin{itemize}
\item The \indexclauseoption{default}{shared} clause means that all
  variables from the outer scope are shared in the parallel region; any private variables
  need to be declared explicitly. This is the default behavior.
\item The \indexclauseoption{default}{private} clause means that all
  outer variables become private in the parallel region. They are not
  initialized; see the next option. Any shared variables in the
  parallel region
  need to be declared explicitly. This value is not available in~C.
\item The \indexclauseoption{default}{firstprivate} clause means all
  outer variables are private in the parallel region, and initialized
  with their outer value.  Any shared variables
  need to be declared explicitly. This value is not available in~C.
\item The \indexclauseoption{default}{none} option is good for debugging, 
  because it forces you to specify for each variable in the parallel region
  whether it's private or shared. Also,
  if your code
behaves differently in parallel from sequential there is probably a data race.
Specifying the status of every variable is a good way to
debug this.
\end{itemize}

\Level 0 {Array data}
\label{sec:omp-array}

The rules for arrays are slightly different from those for scalar data:
\begin{enumerate}
\item Statically allocated data, that is with a syntax like
\begin{lstlisting}
int array[100];
integer,dimension(:) :: array(100}
\end{lstlisting}
can be shared or private, depending on the clause you use.
\item Dynamically allocated data, that is, created with
  \n{malloc}\index{malloc!and private/shared data} or
  \n{allocate}\index{allocate!and private/shared data}, can only be shared.
\end{enumerate}
Example of the first type: in
%
\cverbatimsnippet[examples/omp/c/alloc.c]{privatearray}
%
each thread gets a private copy of the array, properly initialized.

On the other hand, in
%
\cverbatimsnippet[examples/omp/c/alloc.c]{privatepointer}
%
each thread gets a private pointer, but all pointers point to the same
object.

\Level 0 {First and last private}

Above, you saw that private variables are completely separate from any
variables by the same name in the surrounding scope. However, there
are two cases where you may want some \indexterm{storage association}
between a private variable and a global counterpart.

First of all, private variables are created with an undefined value.
You can force their initialization with \indexclause{firstprivate}.
\begin{lstlisting}
  int t=2;
#pragma omp parallel firstprivate(t)
  {
    t += f( omp_get_thread_num() );
    g(t);
  }
\end{lstlisting}
The variable \n{t} behaves like a private variable, except that it is
initialized to the outside value.

Secondly, you may want a private value to be preserved to the
environment outside the parallel region. This really only makes sense
in one case, where you
preserve a private variable
from the last iteration of a parallel loop, or the last section in an
\indexpragma{sections} construct. This is done with \indexclause{lastprivate}:
\begin{lstlisting}
#pragma omp parallel for \
        lastprivate(tmp)
for (i=0; i<N; i+) {
  tmp = ......
  x[i] = .... tmp ....
}
..... tmp ....
\end{lstlisting}

\Level 0 {Persistent data through \texttt{threadprivate}}
\label{sec:threadprivate}

Most data in OpenMP parallel regions is either inherited
from the master thread and therefore shared, or temporary within the scope of the
region and fully private.
There is also a mechanism for \emph{thread-private
  data}\index{thread!private data},
which is not limited in lifetime to one parallel region.
The \indexpragma{threadprivate} pragma is used to declare that each thread
is to have a private copy of a variable:
\begin{lstlisting}
#pragma omp threadprivate(var)
\end{lstlisting}
The variable needs be:
\begin{itemize}
\item a file or static variable in~C,
\item a static class member in~C++, or
\item a program variable or common block in Fortran.
\end{itemize}

\Level 1 {Thread private initialization}

If each thread needs a different value in its threadprivate variable,
the initialization needs to happen in a parallel region.

In the following example a team of 7 threads is created, all of which
set their thread-private variable. Later, this variable is read by a
larger team: the variables that have not been set are undefined,
though often simply zero:
%
\cverbatimsnippet[examples/omp/c/threadprivate.c]{threadprivate}

\begin{fortrannote}{Private common blocks}
  Named common blocks can be made thread-private with the syntax
\begin{lstlisting}
$!OMP threadprivate( /blockname/ )
\end{lstlisting}
Example:
%
\fsnippetwithoutput{threadprivf}{examples/omp/f}{priv}
%
\end{fortrannote}

On the other hand, if the thread private data starts out identical in
all threads, the \indexclause{copyin} clause can be used:
\begin{lstlisting}
#pragma omp threadprivate(private_var)

private_var = 1;
#pragma omp parallel copyin(private_var)
  private_var += omp_get_thread_num()
\end{lstlisting}

If one thread needs to set all thread private data to its value, the
\indexclause{copyprivate} clause can be used:
\begin{lstlisting}
#pragma omp parallel
{
  ...
#pragma omp single copyprivate(private_var)
  private_var = read_data();
  ...
}
\end{lstlisting}

Threadprivate variables require \indexompshow{OMP_DYNAMIC} to be
switched off.

\Level 1 {Thread private example}

The typical application for thread-private variables is in
\indexterm{random number generator}s.
A random number generator needs saved state, since it computes each next value
from the current one. To have a parallel generator, each thread will create
and initialize a private `current value' variable. This will persist
even when the execution is not in a parallel region; it gets updated only
in a parallel region.

\begin{exercise}
  \label{ex:random-mandel}
  Calculate the area of the \indexterm{Mandelbrot set} by random
  sampling. Initialize the random number generator separately for each
  thread; then use a parallel loop to evaluate the points.
  Explore performance implications of the different loop scheduling strategies.
\end{exercise}

\Level 0 {Allocators}

The OpenMP was initially designed for shared memory.
With accelerators (see chapter~\ref{ch:omp-gpu}, non-coherent memory was added to this.
In the \ompstandard{5} standard, the story is further complicated, to account
for new memory types such as \indextermsub{high-bandwidth}{memory}
and \indextermsub{non-volatile}{memory}.

There are several ways of using the OpenMP memory allocators.
\begin{itemize}
\item
  First, in a directory on a static array:
\begin{lstlisting}
float A[N], B[N];
#pragma omp allocate(A) \
    allocator(omp_large_cap_mem_alloc)
\end{lstlisting}
\item As a clause on private variables:
\begin{lstlisting}
#pragma omp task private(B) allocate(omp_const_mem_alloc: B)
\end{lstlisting}
\item
  With \indexompshow{omp_alloc}, using a (possibly user-defined) allocator.
\end{itemize}

Next, there are memory spaces. The binding between OpenMP identifiers and hardware
is implementation defined.

\Level 1 {Pre-defined types}

Allocators:
\indexompshow{omp_default_mem_alloc},
\indexompshow{omp_large_cap_mem_alloc},
\indexompshow{omp_const_mem_alloc},
\indexompshow{omp_high_bw_mem_alloc},
\indexompshow{omp_low_lat_mem_alloc},
\indexompshow{omp_cgroup_mem_alloc},
\indexompshow{omp_pteam_mem_alloc},
\indexompshow{omp_thread_mem_alloc}.

Memory spaces:
\indexompshow{omp_default_mem_space},
\indexompshow{omp_large_cap_mem_space},
\indexompshow{omp_const_mem_space},
\indexompshow{omp_high_bw_mem_space},
\indexompshow{omp_low_lat_mem_space}.

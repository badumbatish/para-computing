% -*- latex -*-
%%%%%%%%%%%%%%%%%%%%%%%%%%%%%%%%%%%%%%%%%%%%%%%%%%%%%%%%%%%%%%%%
%%%%%%%%%%%%%%%%%%%%%%%%%%%%%%%%%%%%%%%%%%%%%%%%%%%%%%%%%%%%%%%%
%%%%
%%%% This text file is part of the lecture slides for
%%%% `Parallel Programming in MPI and OpenMP'
%%%% by Victor Eijkhout, copyright 2012-9
%%%%
%%%% ompreview.tex : OpenMP concepts review
%%%%
%%%%%%%%%%%%%%%%%%%%%%%%%%%%%%%%%%%%%%%%%%%%%%%%%%%%%%%%%%%%%%%%
%%%%%%%%%%%%%%%%%%%%%%%%%%%%%%%%%%%%%%%%%%%%%%%%%%%%%%%%%%%%%%%%

\Level 0 {Concepts review}

\begin{multicols}{2}
\Level 1 {Basic concepts}

\begin{itemize}
\item
process / thread / thread team
\item
threads / cores / tasks
\item
directives / library functions / environment variables
\end{itemize}

\Level 1 {Parallel regions}

execution by a team

\Level 1 {Work sharing}

\begin{itemize}
\item
loop / sections / single / workshare
\item
implied barrier
\item
loop scheduling, reduction
\item
sections
\item
single vs master
\item
(F) workshare
\end{itemize}

\Level 1 {Data scope}

\begin{itemize}
\item
shared vs private, C vs F
\item
loop variables and reduction variables
\item
default declaration
\item
firstprivate, lastprivate
\end{itemize}

\Level 1 {Synchronization}

\begin{itemize}
\item
barriers, implied and explicit
\item
nowait
\item
critical sections
\item
locks, difference with critical
\end{itemize}

\Level 1 {Tasks}

\begin{itemize}
\item
generation vs execution
\item
dependencies
\end{itemize}


\end{multicols}

\pagebreak

\Level 0 {Review questions}

\Level 1 {Directives}

What do the following program output?
\lstset{frame=single} %pyskip

\lstset{language=C} %pyskip
\begin{multicols}{2}
\small
\begin{lstlisting}
int main() {
  printf("procs %d\n",
    omp_get_num_procs());
  printf("threads %d\n",
    omp_get_num_threads());
  printf("num %d\n",
    omp_get_thread_num());
  return 0;
}
\end{lstlisting}
\columnbreak

\begin{lstlisting}
int main() {
#pragma omp parallel
  {
  printf("procs %d\n",
    omp_get_num_procs());
  printf("threads %d\n",
    omp_get_num_threads());
  printf("num %d\n",
    omp_get_thread_num());
  }
  return 0;
}
\end{lstlisting}
\end{multicols}

\lstset{language=Fortran} %pyskip
\begin{multicols}{2}
\small
\begin{lstlisting}
Program main
  use omp_lib
  print *,"Procs:",&
    omp_get_num_procs()
  print *,"Threads:",&
    omp_get_num_threads()
  print *,"Num:",&
    omp_get_thread_num()
End Program
\end{lstlisting}
\columnbreak

\begin{lstlisting}
Program main
  use omp_lib
!$OMP parallel
  print *,"Procs:",&
    omp_get_num_procs()
  print *,"Threads:",&
    omp_get_num_threads()
  print *,"Num:",&
    omp_get_thread_num()
!$OMP end parallel
End Program
\end{lstlisting}
\end{multicols}

\vfill\pagebreak

\Level 1 {Parallelism}

Can the following loops be parallelized? If so, how?
(Assume that all arrays are already filled in,
and that there are no out-of-bounds errors.) 
\lstset{language=C} %pyskip
\begin{multicols}{2}
\small
\begin{lstlisting}
// variant #1
for (i=0; i<N; i++) {
  x[i] = a[i]+b[i+1];
  a[i] = 2*x[i] + c[i+1];
}
\end{lstlisting}

\begin{lstlisting}
// variant #2
for (i=0; i<N; i++) {
  x[i] = a[i]+b[i+1];
  a[i] = 2*x[i+1] + c[i+1];
}
\end{lstlisting}
\columnbreak

\begin{lstlisting}
// variant #3
for (i=1; i<N; i++) {
  x[i] = a[i]+b[i+1];
  a[i] = 2*x[i-1] + c[i+1];
}
\end{lstlisting}

\begin{lstlisting}
// variant #4
for (i=1; i<N; i++) {
  x[i] = a[i]+b[i+1];
  a[i+1] = 2*x[i-1] + c[i+1];
}
\end{lstlisting}
\end{multicols}

\lstset{language=Fortran} %pyskip
\begin{multicols}{2}
\small
\begin{lstlisting}
! variant #1
do i=1,N
  x(i) = a(i)+b(i+1)
  a(i) = 2*x(i) + c(i+1)
end do
\end{lstlisting}

\begin{lstlisting}
! variant #2
do i=1,N
  x(i) = a(i)+b(i+1)
  a(i) = 2*x(i+1) + c(i+1)
end do
\end{lstlisting}
\columnbreak

\begin{lstlisting}
! variant #3
do i=2,N
  x(i) = a(i)+b(i+1)
  a(i) = 2*x(i-1) + c(i+1)
end do
\end{lstlisting}

\begin{lstlisting}
! variant #3
do i=2,N
  x(i) = a(i)+b(i+1)
  a(i+1) = 2*x(i-1) + c(i+1)
end do
\end{lstlisting}
\end{multicols}

\vfill\pagebreak

\Level 1 {Data and synchronization}

\Level 2 {}

What is the output of the following fragments? Assume that there are four threads.

\lstset{language=C} %pyskip
\begin{multicols}{2}
\small
\begin{lstlisting}
// variant #1
int nt;
#pragma omp parallel
  {
  nt = omp_get_thread_num();
  printf("thread number: %d\n",nt);
  }
\end{lstlisting}

\begin{lstlisting}
// variant #2
int nt;
#pragma omp parallel private(nt)
  {
  nt = omp_get_thread_num();
  printf("thread number: %d\n",nt);
  }
\end{lstlisting}

\begin{lstlisting}
// variant #3
int nt;
#pragma omp parallel
  {
#pragma omp single
    {
    nt = omp_get_thread_num();
    printf("thread number: %d\n",nt);
    }
  }
\end{lstlisting}
\columnbreak

\begin{lstlisting}
// variant #4
int nt;
#pragma omp parallel
  {
#pragma omp master
    {
    nt = omp_get_thread_num();
    printf("thread number: %d\n",nt);
    }
  }
\end{lstlisting}

\begin{lstlisting}
// variant #5
int nt;
#pragma omp parallel
  {
#pragma omp critical
    {
    nt = omp_get_thread_num();
    printf("thread number: %d\n",nt);
    }
  }
\end{lstlisting}
\end{multicols}

\lstset{language=Fortran} %pyskip
\begin{multicols}{2}
\small
\begin{lstlisting}
! variant #1
  integer nt
!$OMP parallel
  nt = omp_get_thread_num()
  print *,"thread number:",nt
!$OMP end parallel
\end{lstlisting}

\begin{lstlisting}
! variant #2
  integer nt
!$OMP parallel private(nt)
  nt = omp_get_thread_num()
  print *,"thread number:",nt
!$OMP end parallel
\end{lstlisting}

\begin{lstlisting}
! variant #3
  integer nt
!$OMP parallel
!$OMP single
    nt = omp_get_thread_num()
    print *,"thread number:",nt
!$OMP end single
!$OMP end parallel
\end{lstlisting}
\columnbreak

\begin{lstlisting}
! variant #4
  integer nt
!$OMP parallel
!$OMP master
    nt = omp_get_thread_num()
    print *,"thread number:",nt
!$OMP end master
!$OMP end parallel
\end{lstlisting}

\begin{lstlisting}
! variant #5
  integer nt
!$OMP parallel
!$OMP critical
    nt = omp_get_thread_num()
    print *,"thread number:",nt
!$OMP end critical
!$OMP end parallel
\end{lstlisting}
\end{multicols}

\Level 2 {}

The following is an attempt to parallelize a serial code.
Assume that all variables and arrays are defined.
What errors and potential problems do you see in this code? How would you fix them?

\begin{multicols}{2}
\lstset{language=C} %pyskip
\small
\begin{lstlisting}
#pragma omp parallel
{
  x = f();
  #pragma omp for
  for (i=0; i<N; i++)
    y[i] = g(x,i);
  z = h(y);
}
\end{lstlisting}
\columnbreak

\lstset{language=Fortran} %pyskip
\begin{lstlisting}
!$OMP parallel
  x = f()
!$OMP do
  do i=1,N
    y(i) = g(x,i)
  end do
!$OMP end do 
  z = h(y)
!$OMP end parallel
\end{lstlisting}
\end{multicols}


\vfill\pagebreak

\Level 2 {}

Assume two threads. What does the following program output?

\begin{lstlisting}
int a;
#pragma omp parallel private(a) {
  ...
  a = 0;
  #pragma omp for
  for (int i = 0; i < 10; i++)
  {
    #pragma omp atomic
    a++; }
  #pragma omp single
    printf("a=%e\n",a);
}
\end{lstlisting}

\Level 1 {Reductions}

\Level 2 {}

Is the following code correct? Is it efficient? If not, can you improve it?
\begin{verbatim}
#pragma omp parallel shared(r)
{
  int x;
  x = f(omp_get_thread_num());
#pragma omp critical
  r += f(x);
}
\end{verbatim}

\Level 2 {}

Compare two fragments:
\begin{multicols}{2}
\lstset{language=C}
\begin{lstlisting}
// variant 1
#pragma omp parallel reduction(+:s)
#pragma omp for
  for (i=0; i<N; i++)
    s += f(i);
\end{lstlisting}
\columnbreak
\begin{lstlisting}
// variant 2
#pragma omp parallel 
#pragma omp for reduction(+:s)
  for (i=0; i<N; i++)
    s += f(i);
\end{lstlisting}
\end{multicols}

\begin{multicols}{2}
\lstset{language=Fortran}
\begin{lstlisting}
! variant 1
!$OMP parallel reduction(+:s)
!$OMP do
  do i=1,N
    s += f(i);
  end do
!$OMP end do
!$OMP end parallel 
\end{lstlisting}
\columnbreak
\begin{lstlisting}
! variant 2
!$OMP parallel 
!$OMP do reduction(+:s)
  do i=1,N
    s += f(i);
  end do
!$OMP end do
!$OMP end parallel 
\end{lstlisting}
\end{multicols}

Do they compute the same thing?

\vfill\pagebreak

\Level 1 {Barriers}

Are the following two code fragments well defined?
\begin{multicols}{2}
\lstset{language=C}
\begin{lstlisting}
#pragma omp parallel 
{
#pragma omp for
for (mytid=0; mytid<nthreads; mytid++)
  x[mytid] = some_calculation();
#pragma omp for
for (mytid=0; mytid<nthreads-1; mytid++)
  y[mytid] = x[mytid]+x[mytid+1];
}
\end{lstlisting}
\columnbreak
\begin{lstlisting}
#pragma omp parallel 
{
#pragma omp for
for (mytid=0; mytid<nthreads; mytid++)
  x[mytid] = some_calculation();
#pragma omp for nowait
for (mytid=0; mytid<nthreads-1; mytid++)
  y[mytid] = x[mytid]+x[mytid+1];
}
\end{lstlisting}
\end{multicols}

\Level 1 {Data scope}

The following program is supposed to initialize as many
rows of the array as there are threads.

\begin{multicols}{2}
\small
\lstset{language=C}
\begin{lstlisting}
int main() {
  int i,icount,iarray[100][100];
  icount = -1;
#pragma omp parallel private(i)
  {
#pragma omp critical
    { icount++; }
    for (i=0; i<100; i++) 
      iarray[icount][i] = 1;
  }
  return 0;
}
\end{lstlisting}
\columnbreak
\lstset{language=Fortran}
\begin{lstlisting}
Program main
  integer :: i,icount,iarray(100,100)
  icount = 0
!$OMP parallel private(i)
!$OMP critical
    icount = icount + 1
!$OMP end critical
    do i=1,100
      iarray(icount,i) = 1
    end do
!$OMP end parallel
End program
\end{lstlisting}
\end{multicols}

Describe the behaviour of the program, with argumentation,
\begin{itemize}
\item as given;
\item if you add a clause \n{private(icount)} to the \n{parallel} directive;
\item if you add a clause \n{firstprivate(icount)}.
\end{itemize}

What do you think of this solution:
\begin{multicols}{2}
\small
\lstset{language=C}
\begin{lstlisting}
#pragma omp parallel private(i) shared(icount)
  {
#pragma omp critical
    { icount++;
      for (i=0; i<100; i++) 
        iarray[icount][i] = 1;
    }
  }
  return 0;
}
\end{lstlisting}
\columnbreak

\lstset{language=Fortran}
\begin{lstlisting}
!$OMP parallel private(i) shared(icount)
!$OMP critical
    icount = icount+1
    do i=1,100
      iarray(icount,i) = 1
    end do
!$OMP critical
!$OMP end parallel
\end{lstlisting}
\end{multicols}

\Level 1 {Tasks}

Fix two things in the following example:
\begin{multicols}{2}
\small
\lstset{language=C}
\begin{lstlisting}
#pragma omp parallel
#pragma omp single
{
  int x,y,z;
#pragma omp task
  x = f();
#pragma omp task
  y = g();
#pragma omp task
  z = h();
  printf("sum=%d\n",x+y+z);
}
\end{lstlisting}
\columnbreak
\lstset{language=Fortran}
\begin{lstlisting}
  integer :: x,y,z
!$OMP parallel
!$OMP single

!$OMP task
  x = f()
!$OMP end task

!$OMP task
  y = g()
!$OMP end task

!$OMP task
  z = h()
!$OMP end task

  print *,"sum=",x+y+z
!$OMP end single
!$OMP end parallel
\end{lstlisting}
\end{multicols}


\Level 1 {Scheduling}

Compare these two fragments. Do they compute the same result? What can you say about their efficiency?
\begin{multicols}{2}
\begin{lstlisting}
#pragma omp parallel
#pragma omp single
  {
    for (i=0; i<N; i++) {
    #pragma omp task
      x[i] = f(i)
    }
    #pragma omp taskwait
  }
\end{lstlisting}
\columnbreak
\begin{lstlisting}
#pragma omp parallel
#pragma omp for schedule(dynamic)
  {
    for (i=0; i<N; i++) {
      x[i] = f(i)
    }
  }
\end{lstlisting}
\end{multicols}

How would you make the second loop more efficient?
Can you do something similar for the first loop?


% -*- latex
%%%%%%%%%%%%%%%%%%%%%%%%%%%%%%%%%%%%%%%%%%%%%%%%%%%%%%%%%%%%%%%%
%%%%%%%%%%%%%%%%%%%%%%%%%%%%%%%%%%%%%%%%%%%%%%%%%%%%%%%%%%%%%%%%
%%%%
%%%% This text file is part of the source of 
%%%% `Introduction to High-Performance Scientific Computing'
%%%% by Victor Eijkhout, copyright 2012-2021
%%%%
%%%% This book is distributed under a Creative Commons Attribution 3.0
%%%% Unported (CC BY 3.0) license and made possible by funding from
%%%% The Saylor Foundation \url{http://www.saylor.org}.
%%%%
%%%%
%%%%%%%%%%%%%%%%%%%%%%%%%%%%%%%%%%%%%%%%%%%%%%%%%%%%%%%%%%%%%%%%
%%%%%%%%%%%%%%%%%%%%%%%%%%%%%%%%%%%%%%%%%%%%%%%%%%%%%%%%%%%%%%%%

\newcommand\furtherreading{\Level 0 {Further Reading}\label{sec:furtherreading-\chapshortname}}
\newcommand\heading[1]{\paragraph*{\textbf{#1}}}

{\catcode`\^^I=13 \globaldefs=1
 \newcommand\listing[2]{\begingroup\small\par\vspace{1ex}
  \catcode`\^^I=13 \def^^I{\leavevmode\hspace{40pt}}
  \noindent\fbox{#1}
  \verbatiminput{#2}\endgroup}
 \newcommand\codelisting[1]{\begingroup\small\par\vspace{1ex}
  \catcode`\^^I=13 \def^^I{\leavevmode\hspace{40pt}}
  \noindent\fbox{#1}
  \verbatiminput{#1}\endgroup}
}
\newcommand\inv{^{-1}}\newcommand\invt{^{-t}}
\newcommand\setspan[1]{[\![#1]\!]}
\newcommand\fp[2]{#1\cdot10^{#2}}
\newcommand\fxp[2]{\langle #1,#2\rangle}
\def\n#{\bgroup \catcode`\$=12 \catcode`\_=12 \catcode`\>=12 \catcode`\<=12 \catcode`\#=12
  \catcode`\&=12 \catcode`\^=12 \catcode`\~=12 \def\\{\char`\\}\relax
  \tt \let\next=}

%%
%% use beamer buttons for links to exercise slides
%%
\newcommand\exref[1]{%
  % split at the colon and attach a colon to catch the label of the exercise
  \exrefcolon#1:}
\def\exrefcolon ex:#1:{%
  % we use \label{ex:foo} and \frame[label=exfoo]
  \hyperlink{#1}{\noexpand\beamergotobutton{ex#1}}%
}
\newcommand\exercisesource[1]{}

\newcommand\diag{\mathop{\mathrm {diag}}}
\newcommand\argmin{\mathop{\mathrm {argmin}}}
\newcommand\defined{
  \mathrel{\lower 5pt \hbox{${\equiv\atop\mathrm{\scriptstyle D}}$}}}

\newcommand\bbP{\mathbb{P}}
\newcommand\bbR{\mathbb{R}}

\newtheorem{remark}{Remark}
\expandafter\ifx\csname definition\endcsname\relax
    \newtheorem{definition}{Definition}
\fi
\expandafter\ifx\csname theorem\endcsname\relax
    \newtheorem{theorem}{Theorem}
\fi
\expandafter\ifx\csname lemma\endcsname\relax
    \newtheorem{lemma}{Lemma}
\fi

%%%%
%%%% Verbatim source handling
%%%%

% each chapter has a list of sources
\newtoks\chaptersourcelist
\newcommand\addchaptersource[1]{
  \edef\temp{\global\chaptersourcelist={\the\chaptersourcelist #1}}\temp
}
\newcommand\listchaptersources{
  \expandafter\ChapterSourceHeader\the\chaptersourcelist\LSR
  %\tracingmacros=2 \tracingonline=1
  %\texttt{\the\chaptersourcelist}\par
  \expandafter\ListSourcesRecursively\the\chaptersourcelist\LSR
}
\def\LSR{\LSR}
\def\ChapterSourceHeader#1\LSR{
  \def\test{#1\LSR}
  \ifx\test\LSR
  \else
    \Level 0 {Sources used in this chapter}
  \fi
}
\def\ListSourcesRecursively#1{
  \def\test{#1}
  \ifx\test\LSR
  \else
    % list the file
    \textbf{Listing of code #1}:
    {\footnotesize \verbatiminput{#1}}
    \par
    % continue
    \expandafter\ListSourcesRecursively
  \fi
}

%%%%
%%%% snippets
%%%%
\def\verbatimsnippet#1{\verbatiminput{#1}} % snippets/
\newcommand{\cverbatimsnippet}[2][XX]{
  % record this file as bracketed name
  \addchaptersource{{#1}}
  % typeset as nice C code
  \lstset{style=reviewcode,language=C}\lstinputlisting{#2}}
\newcommand{\cxxverbatimsnippet}[2][XX]{
  % record this file as bracketed name
  \addchaptersource{{#1}}
  % typeset as nice C code
  \lstset{style=reviewcode,language=C++}\lstinputlisting{#2}}
\def\fverbatimsnippet#1{
  \lstset{style=reviewcode,language=Fortran}\lstinputlisting{#1}
  \lstset{style=reviewcode,language=C}
}
\def\pverbatimsnippet#1{
  \lstset{style=reviewcode,language=Python}\lstinputlisting{#1}
  \lstset{style=reviewcode,language=C}
}
\newenvironment{clisting}
    {\lstset{style=reviewcode,language=C}\begin{lstlisting}}
    {\end{lstlisting}
}
\newenvironment{cxxlisting}
    {\lstset{style=reviewcode,language=C++}\begin{lstlisting}}
    {\end{lstlisting}
}

%%%%
%%%% Environments
%%%%
\newcounter{slidecount}
\newenvironment
    {numberedframe}[1]
    {\refstepcounter{slidecount}
      \begin{frame}[containsverbatim]{\arabic{slidecount}\ #1}}
    {\end{frame}}
\newenvironment{question}{\begin{quotation}\textbf{Question.\ }}{\end{quotation}}
\newenvironment{fortrannote}
  {\begin{quotation}\noindent\textsl{Fortran note.\kern1em}\ignorespaces}
  {\end{quotation}}
\newenvironment{pythonnote}
  {\begin{quotation}\noindent\textsl{Python note.\kern1em}\ignorespaces}
  {\end{quotation}}
\newenvironment{taccnote}
  {\begin{quotation}\noindent\textsl{TACC note.\kern1em}\ignorespaces}
  {\end{quotation}}
\newenvironment{mpifour}
  {\begin{quotation}\textbf{MPI-4:\ }\ignorespaces}{\end{quotation}}

%% \newenvironment{highermath}
%%     {\bigskip\begin{quotation}\noindent\emph{MMM}}
%%     {\end{quotation}\bigskip\noindent\ignorespaces}

\usepackage{acronym}
\newwrite\acrowrite
\openout\acrowrite=acronyms.tex
\def\acroitem#1#2{\acrodef{#1}{#2}
    \write\acrowrite{\message{defining #1}\noexpand\acitem{#1}{#2}}
}
\acroitem{AVX}{Advanced Vector Extensions}
\acroitem{BSP}{Bulk Synchronous Parallel}
\acroitem{CAF}{Co-array Fortran}
\acroitem{CUDA}{Compute-Unified Device Architecture}
\acroitem{DAG}{Directed Acyclic Graph}
\acroitem{DSP}{Digital Signal Processing}
\acroitem{FPU}{Floating Point Unit}
\acroitem{FFT}{Fast Fourier Transform}
\acroitem{FSA}{Finite State Automaton}
\acroitem{GPU}{Graphics Processing Unit}
\acroitem{HPC}{High-Performance Computing}
\acroitem{HPF}{High Performance Fortran}
\acroitem{ICV}{Internal Control Variable}
\acroitem{MIC}{Many Integrated Cores}
\acroitem{MPMD}{Multiple Program Multiple Data}
\acroitem{MIMD}{Multiple Instruction Multiple Data}
\acroitem{MPI}{Message Passing Interface}
\acroitem{MTA}{Multi-Threaded Architecture}
\acroitem{NUMA}{Non-Uniform Memory Access}
\acroitem{OS}{Operating System}
\acroitem{PGAS}{Partitioned Global Address Space}
\acroitem{PDE}{Partial Diffential Equation}
\acroitem{PRAM}{Parallel Random Access Machine}
\acroitem{RDMA}{Remote Direct Memory Access}
\acroitem{RMA}{Remote Memory Access}
\acroitem{SAN}{Storage Area Network}
\acroitem{SaaS}{Software as-a Service}
\acroitem{SFC}{Space-Filling Curve}
\acroitem{SIMD}{Single Instruction Multiple Data}
\acroitem{SIMT}{Single Instruction Multiple Thread}
\acroitem{SM}{Streaming Multiprocessor}
\acroitem{SMP}{Symmetric Multi Processing}
\acroitem{SOR}{Successive Over-Relaxation}
\acroitem{SP}{Streaming Processor}
\acroitem{SPMD}{Single Program Multiple Data}
\acroitem{SPD}{symmetric positive definite}
\acroitem{SSE}{SIMD Streaming Extensions}
\acroitem{TLB}{Translation Look-aside Buffer}
\acroitem{UMA}{Uniform Memory Access}
\acroitem{UPC}{Unified Parallel C}
\acroitem{WAN}{Wide Area Network}
\acresetall
\closeout\acrowrite


\def\chaptertitle{\csname\chaptername title\endcsname}
\def\chaptershorttitle{\csname\chaptername shorttitle\endcsname}

%%%%
%%%% index macros without index
%%%%

\newif\ifShowRoutine
\def\indexmpishow#{\bgroup
  \catcode`\_=12 \catcode`\>=12 \catcode`\<=12
  \catcode`\&=12 \catcode`\^=12 \catcode`\~=12
  \def\\{\char`\\}\relax
  \ShowRoutinetrue
  %\tt
  \afterassignment\mpitoindex\edef\indexedmpi}
\def\mpitoindex{%\tracingmacros=2
  \edef\tmp{%
    \noexpand\ifShowRoutine
        \noexpand\lstinline+\indexedmpi+\noexpand\nobreak
    \noexpand\fi
    %\noexpand\index{\indexedmpi@{\catcode95=12 \noexpand\texttt{\indexedmpi}}}%
  }%
  \tmp
  \egroup\nobreak
}
\let\indexmpiref\indexmpishow
\let\indexmpidef\indexmpishow
\let\indexmpldef\indexmpishow
\let\indexmplref\indexmpishow
\let\indexmplshow\indexmpishow
\let\indexpetscshow\indexmpishow

\newenvironment
    {mplimpl}
    {\begin{quotation}\textsl{MPL implementation note:}\ }
    {\end{quotation}}

\newcommand\petscroutineslide[1]{
  \begin{frame}[containsverbatim]{#1}
    \footnotesize
    \verbatiminput{#1}
  \end{frame}
}

% -*- latex -*-
%%%%%%%%%%%%%%%%%%%%%%%%%%%%%%%%%%%%%%%%%%%%%%%%%%%%%%%%%%%%%%%%
%%%%%%%%%%%%%%%%%%%%%%%%%%%%%%%%%%%%%%%%%%%%%%%%%%%%%%%%%%%%%%%%
%%%%
%%%% This LaTeX file is part of the source of 
%%%% `Parallel Computing'
%%%% by Victor Eijkhout, copyright 2012-2021
%%%%
%%%% idxmacs.tex : index-related macros.
%%%%
%%%%%%%%%%%%%%%%%%%%%%%%%%%%%%%%%%%%%%%%%%%%%%%%%%%%%%%%%%%%%%%%
%%%%%%%%%%%%%%%%%%%%%%%%%%%%%%%%%%%%%%%%%%%%%%%%%%%%%%%%%%%%%%%%

\usepackage[original]{imakeidx}

\makeindex
\makeindex[name=mpi]
\makeindex[name=mpl]
\makeindex[name=sycl]
\makeindex[name=omp]
\makeindex[name=petsc]
\makeindex[name=python]

\newcommand{\indextermp}[1]{\emph{#1s}\index{#1}}
\newcommand{\indextermsub}[2]{\emph{#1 #2}\index{#2!#1}}
\newcommand{\indextermsubh}[2]{\emph{#1-#2}\index{#2!#1}}
\newcommand{\indextermsubdef}[2]{\emph{#1 #2}\index{#2!#1|textbf}}
\newcommand{\indextermsubdefh}[2]{\emph{#1-#2}\index{#2!#1|textbf}}
\newcommand{\indextermsubp}[2]{\emph{#1 #2s}\index{#2!#1}}
\newcommand{\indextermbus}[2]{\emph{#1 #2}\index{#1!#2}}
\newcommand{\indextermbusp}[2]{\emph{#1 #2s}\index{#1!#2}}
\newcommand{\indextermbusdef}[2]{\emph{#1 #2}\index{#1!#2|textbf}}
\newcommand{\indextermstart}[1]{\emph{#1}\index{#1|(}}
\newcommand{\indextermend}[1]{\index{#1|)}}
\newcommand{\indexstart}[1]{\index{#1|(}}
\newcommand{\indexend}[1]{\index{#1|)}}
\makeatletter
\newcommand\indexac[1]{\emph{\ac{#1}}%
  %\tracingmacros=2 \tracingcommands=2
  \edef\tmp{\noexpand\index{%
    \expandafter\expandafter\expandafter
        \@secondoftwo\csname fn@#1\endcsname%
    @\acl{#1} (#1)}}\tmp}
\newcommand\indexacp[1]{\emph{\ac{#1}}%
  %\tracingmacros=2 \tracingcommands=2
  \edef\tmp{\noexpand\index{%
    \expandafter\expandafter\expandafter
        \@secondoftwo\csname fn@#1\endcsname%
    @\acl{#1} (#1)}}\tmp}
\newcommand\indexacf[1]{\emph{\acf{#1}}%
  \edef\tmp{\noexpand\index{%
    \expandafter\expandafter\expandafter
        \@secondoftwo\csname fn@#1\endcsname
    @\acl{#1} (#1)}}\tmp}
\newcommand\indexacstart[1]{%
  \edef\tmp{\noexpand\index{%
    \expandafter\expandafter\expandafter
        \@secondoftwo\csname fn@#1\endcsname
    @\acl{#1} (#1)|(}}\tmp}
\newcommand\indexacend[1]{%
  \edef\tmp{\noexpand\index{%
    \expandafter\expandafter\expandafter
        \@secondoftwo\csname fn@#1\endcsname
    @\acl{#1} (#1)|)}}\tmp}
\makeatother

\newif\ifShowRoutine
{ \catcode`\_=13
  %% MPI index routines
  %% \gdef\indexmpishow#{\bgroup
  %%   \InnocentChars \ShowRoutinetrue
  %%   \afterassignment\mpitoindex\edef\indexedmpi}
  %% \gdef\indexmplshow#{\bgroup %% MPL variant
  %%   \InnocentChars \ShowRoutinetrue
  %%   \afterassignment\mpltoindex\edef\indexedmpi}
  \gdef\indexmpishowf#{\bgroup
    \InnocentChars \def\\{\char`\\}\relax
    \ShowRoutinetrue
    \afterassignment\mpitoindexf\edef\indexedmpi}
  \gdef\indexmpishowsub#{\bgroup
    \InnocentChars
    \catcode`\_=13 \def_{\char95\discretionary{}{}{}}
    \def\\{\char`\\}\relax
    %
    \tt \afterassignment\defineindexmpisubtwo \xdef\indexedmpisubone}
  \gdef\defineindexmpisubtwo{\afterassignment\mpitoindexsub \gdef\indexedmpisubtwo}
}

%%%%
%%%% Stuff to index
%%%%
\def\indexgenshow#1#{\bgroup
    \InnocentChars \ShowRoutinetrue
    \afterassignment#1\edef\indexedmpi
}
% mpi
\newcommand\mpitoindex{\gentoindex{mpi}{textrm}}
\newcommand\mpitoindexbf{\gentoindex{mpi}{textbf}}
\newcommand\mpitoindexit{\gentoindex{mpi}{textsl}}
% mpl
\newcommand\mpltoindex{\gentoindex{mpl}{textrm}}
\newcommand\mpltoindexbf{\gentoindex{mpl}{textbf}}
%
\newcommand\omptoindex{\gentoindex{omp}{textrm}}
\newcommand\omptoindexbf{\gentoindex{omp}{textbf}}
\newcommand\omptoindexit{\gentoindex{omp}{textsl}}
% basic routine
\newcommand\gentoindex[2]{%% 1: index name, 2: font cs
  \edef\tmp{%
    \noexpand\ifShowRoutine
        \noexpand\lstinline+\indexedmpi+\noexpand\nobreak
    \noexpand\fi
    \noexpand\index[#1]{\indexedmpi@{\catcode95=12 \noexpand\texttt{\indexedmpi}}|#2}%
  }%
  \tmp
  \egroup\nobreak
}

\def\mpitoindexf{
  \edef\tmp{%
    \noexpand\ifShowRoutine
        \noexpand\lstinline+\indexedmpi+\noexpand\nobreak
    \noexpand\fi
    \noexpand\index[mpi]{\indexedmpi@{\catcode95=12
        \noexpand\texttt{\indexedmpi}}!in Fortran}%
  }%
  \tmp
  \egroup\nobreak
}
%% \def\mpitoindexit{%\tracingmacros=2
%%   \edef\tmp{%
%%     \noexpand\n{\indexedmpi}%
%%     \noexpand\index[mpi]{\indexedmpi@{\catcode95=12 \noexpand\texttt{\indexedmpi}}|textsl}}%
%%   \tmp
%%   \egroup\nobreak
%% }
\def\mpitoindexsub{%\tracingmacros=2
  \edef\tmp{%
    \noexpand\lstinline+\indexedmpisubone+ {\indexedmpisubtwo}%
    \noexpand\index[mpi]{\indexedmpisubone@{\catcode95=12 \noexpand\texttt{\indexedmpisubone}}!\indexedmpisubtwo@{\indexedmpisubtwo}}}%
  \tmp
  \egroup\nobreak
}


%%%%
%%%% OpenMP to index
%%%%

%%%%
%%%% MPI Routine Ref
%%%%

%%%%
%%%% PETSc Routine Ref
%%%%
{ \catcode`\_=13
  \gdef\indexpetsc#{\bgroup
    \InnocentChars
    \ShowRoutinefalse
    \tt \afterassignment\petsctoindex\edef\indexedpetsc}
  \gdef\indexpetscoption#{\bgroup
    \InnocentChars
    \ShowRoutinetrue
    \tt \afterassignment\petscoptiontoindex\edef\indexedpetsc}
  \global\let\indexpetscfile\indexpetscshow
}
\def\petsctoindex{%\tracingmacros=2
  \edef\tmp{%
    \noexpand\lstinline+\indexedpetsc+%
    \noexpand\index[petsc]{\indexedpetsc@{\catcode95=12 \noexpand\texttt{\indexedpetsc}}}}%
  \tmp
  \egroup
}
\def\petscoptiontoindex{%\tracingmacros=2
  \edef\tmp{%
    \hbox{\noexpand\lstinline+-\indexedpetsc+}%
    \noexpand\index[petsc]{-\indexedpetsc@{\catcode95=12 \noexpand\texttt{-\indexedpetsc}}}}%
  \tmp
  \egroup
}
\def\petsctoindexbf{%\tracingmacros=2
  \edef\tmp{%
    \noexpand\lstinline+\indexedpetsc+%
    \noexpand\index[petsc]{\indexedpetsc@{\catcode95=12 \noexpand\texttt{\indexedpetsc}}|textbf}%
  }%
  \tmp
  \egroup
}


%%%%
%%%% routine ref
%%%%

\def\indexpetscxref#1#2{\def\indexedroutine{#1}\def\displayedroutine{#2}%
  %\petscIndexBF -> \RoutineIndexBF{petsc}
  \edef\tmp{%
    \noexpand\lstinline+\indexedroutine+
    (figure\noexpand~\noexpand\ref{ref:\displayedroutine})%
    \noexpand\index[petsc]{%
      \indexedroutine@{\catcode95=12 \noexpand\texttt{\indexedroutine}}|textbf}%
  }%
  \tmp
}  

%% mpi
\newcommand\indexmpishow{\PackageIndex\mpiIndexRM}
\newcommand\indexmpidepr{\PackageIndex\mpiIndexRd}
\newcommand\indexmpidef {\PackageIndex\mpiIndexBF}
\newcommand\indexmpiex  {\PackageIndex\mpiIndexIT}
% versions that generate man page
\newcommand\indexmpiref {\PackageIndex   \mpiIndexAndRef}
\newcommand\indexmpixref{\PackageIndexTwo\mpiIndexAndXRef}
\let\indexmpirep\indexmpishow % need to incorporate link to man page

\newcommand\mpiIndexBF{\RoutineIndexFont{mpi}{textbf}\endgroup}
\newcommand\mpiIndexRM{\RoutineIndexFont{mpi}{textrm}\endgroup}
\newcommand\mpiIndexRd{\RoutineDeprcFont{mpi}{textrm}\endgroup}
\newcommand\mpiIndexIT{\RoutineIndexFont{mpi}{textsl}\endgroup}
\newcommand\mpiIndexAndRef{% this is called through afterassignment; \indexedroutine has been set
  \global\let\referencedroutine\indexedroutine
  \RoutineShowAndIndex{mpi}{textbf}% show name, index, reference to figure
  \endgroup % end of innocent characters
  \def\routinefam{mpi}\RoutineRefDisplay
}
\newcommand\mpiIndexAndXRef{%
  \RoutineShowAndIndex{mpi}{textbf}%
  \endgroup}

% mpl
\newcommand\indexmplshow{\PackageIndex\mplIndexRM}
\newcommand\indexmpldef {\PackageIndex\mplIndexBF}
\newcommand\indexmplref {\PackageIndex\mplIndexAndRef}
\newcommand\indexmplex  {\PackageIndex\mplIndexIT}
\let\indexmplrep\indexmplshow % need to incorporate link to man page

\newcommand\mplIndexBF{\RoutineIndexFont{mpl}{textbf}\endgroup}
\newcommand\mplIndexRM{\RoutineIndexFont{mpl}{textrm}\endgroup}
\newcommand\mplIndexIT{\RoutineIndexFont{mpl}{textsl}\endgroup}
\newcommand\mplIndexAndRef{%
  \let\referencedroutine\indexedroutine
  \RoutineShowAndIndex{mpl}{textbf}%
  \def\routinefam{mpl}\RoutineRefDisplay\endgroup}

% omp
\newcommand\indexompterm  {\PackageIndex\ompIndexTe}
\newcommand\indexompshow  {\PackageIndex\ompIndexRM}
\newcommand\indexompdepr  {\PackageIndex\ompIndexRd}
\newcommand\indexompdef   {\PackageIndex\ompIndexBF}
\newcommand\indexompclause{\PackageIndex\ompIndexRM}
\newcommand\indexompref   {\PackageIndex\ompIndexAndRef}
\newcommand\indexompex    {\PackageIndex\ompIndexIT}
\let\indexomprep\indexompshow % need to incorporate link to man page

\def\indexpragma#1{\texttt{#1}\index[omp]{omp!#1|texttt}}
\def\indexpragmadef#1{\texttt{#1}\index[omp]{omp!#1|textbf}}
\index{pragma|see{for list see under `omp'}}
%\def\indexompclause#1{\texttt{#1}\index[omp]{omp clause!#1}}
\def\indexclause#1{\texttt{#1}\index[omp]{omp clause!#1}}
\def\indexclausedef#1{\texttt{#1}\index[omp]{omp clause!#1|textbf}}
\def\indexclauseoption#1#2{\texttt{#2}\index[omp]{omp clause!#1!#2}}
\def\indexompclauseoption#1#2{\texttt{#2}\index[omp]{omp clause!#1!#2}}

\newcommand\ompIndexTe{\TermIndexFont{omp}{textrm}\endgroup}
\newcommand\ompIndexRM{\RoutineIndexFont{omp}{textrm}\endgroup}
\newcommand\ompIndexBF{\RoutineIndexFont{omp}{textbf}\endgroup}
\newcommand\ompIndexRd{\RoutineDeprcFont{omp}{textrm}\endgroup}
\newcommand\ompIndexIT{\RoutineIndexFont{omp}{textsl}\endgroup}
\newcommand\ompIndexAndRef{%
  \let\referencedroutine\indexedroutine
  \RoutineIndexFont{omp}{textbf}%
  \def\routinefam{omp}\RoutineRefDisplay\endgroup}

% petsc
\newcommand\indexpetscshow{\PackageIndex\petscIndexRM}
\newcommand\indexpetscdepr{\PackageIndex\petscIndexRd}
\newcommand\indexpetscdef {\PackageIndex\petscIndexBF}
\newcommand\indexpetscref {\PackageIndex\petscIndexAndRef}
\newcommand\indexpetscex  {\PackageIndex\petscIndexIT}
\newcommand\indexpetsctt  {\PackageIndex\petscIndexTT}
\newcommand\indexpetscfile{\PackageIndex\petscIndexTT}
\let\indexpetscrep\indexpetscshow % need to incorporate link to man page

\newcommand\petscIndexBF{\RoutineIndexFont{petsc}{textbf}\endgroup}
\newcommand\petscIndexRM{\RoutineIndexFont{petsc}{textrm}\endgroup}
\newcommand\petscIndexRd{\RoutineDeprcFont{petsc}{textrm}\endgroup}
\newcommand\petscIndexIT{\RoutineIndexFont{petsc}{textsl}\endgroup}
\newcommand\petscIndexTT{\RoutineIndexFont{petsc}{texttt}\endgroup}
\newcommand\petscIndexAndRef{%
  \let\referencedroutine\indexedroutine
  \RoutineShowAndIndex{petsc}{textbf}%
  \def\routinefam{petsc}\RoutineRefDisplay\endgroup}

% sycl
\newcommand\indexsyclshow{\PackageIndex\syclIndexRM}
\newcommand\indexsycldef {\PackageIndex\syclIndexBF}
\newcommand\syclIndexBF{\RoutineIndexFont{sycl}{textbf}\endgroup}
\newcommand\syclIndexRM{\RoutineIndexFont{sycl}{textrm}\endgroup}

%%
%% Define `indexedroutine' and invoke #1 which will process this
%%
\newcommand\PackageIndex[1]{\begingroup
    \InnocentChars \ShowRoutinetrue
    \afterassignment#1\xdef\indexedroutine
}

%%
%% Define `indexedroutine' and `referencedroutine' and invoke #1 which will process this
%%
\newcommand\PackageIndexTwo[1]{\begingroup
    \InnocentChars \ShowRoutinetrue
    \let\routineprocess=#1%
    \afterassignment\defineReferencedAndProcess
    \edef\indexedroutine
}
\newcommand\defineReferencedAndProcess{%
  \afterassignment\routineprocess
  \edef\referencedroutine}

%%
%% Write `indexedroutine' to index #1 with #2 font
%%
\def\RoutineIndexFont#1#2{% 
  \def\ifont{#2}\def\bfont{textbf}%
  \edef\tmp{%
    \noexpand\lstinline+\indexedroutine+%
    \ifx\ifont\bfont\noexpand\label{def:\indexedroutine}\fi%
    \noexpand\index[#1]{%
      \indexedroutine@{\catcode95=12 \noexpand\texttt{\indexedroutine}}|#2}%
  }%
  \tmp
}
\def\TermIndexFont#1#2{% 
  \def\ifont{#2}\def\bfont{textbf}%
  \edef\tmp{%
    \noexpand\textsl{\indexedroutine}%
    \ifx\ifont\bfont\noexpand\label{def:\indexedroutine}\fi%
    \noexpand\index[#1]{\indexedroutine|#2}%
    \noexpand\index{\indexedroutine (OpenMP)}%
  }%
  \tmp
}
\def\RoutineDeprcFont#1#2{% 
  \edef\tmp{%
    \noexpand\lstinline+\indexedroutine+%
    \noexpand\index[#1]{%
      \indexedroutine@{\catcode95=12 \noexpand\texttt{\indexedroutine} (deprecated)}|#2}%
  }%
  \tmp
}

%%
%% Write `indexedroutine' to index #1 with #2 font
%% also: generate manpage float
%%
\def\RoutineShowAndIndex#1#2{% write `indexedroutine' to index #1 with #2 font
  \def\ifont{#2}\def\bfont{textbf}%
  \edef\tmp{%
    \noexpand\lstinline+\indexedroutine+%
    \ifx\ifont\bfont\noexpand\label{def:\indexedroutine}\fi%
    \noexpand\index[#1]{%
      \indexedroutine@{\catcode95=12 \noexpand\texttt{\indexedroutine}}|#2}\ \relax
    (figure\noexpand~\noexpand\ref{ref:\referencedroutine})%
  }%
  \tmp%
}

\newcommand\RoutineRefStyle{
  \catcode`\_=12 % 13 \def_{\char95\discretionary{}{}{}}
  \catcode`\>=12 \catcode`\<=12
  \catcode`\&=12 \catcode`\^=12 \catcode`\~=12 \def\\{\char`\\}\relax
}
\newcommand\RoutineIndexStyle{
  \catcode`\_=12 % 13 \def_{\char95\discretionary{}{}{}}
  \catcode`\>=12 \catcode`\<=12
  \catcode`\&=12 \catcode`\^=12 \catcode`\~=12 \def\\{\char`\\}\relax
}
{ \catcode`\_=12
\gdef\underscore{_}
}

\usepackage{newfloat} % ,caption}
\DeclareFloatingEnvironment[fileext=man,placement={tp},name=Manpage]{manpage}
%\captionsetup[manpage]{justification=justified,labelformat=empty}

%\newcounter{manpage}
{ %\catcode`\_=13
  \makeatother
  \gdef\RoutineRefDisplay{%
    \begin{manpage}%
      % counter and reference to counter
      \refstepcounter{manpage}%
      \edef\labeltext{ref:\indexedroutine}
      \edef\tmp{\noexpand\label{\labeltext}}\tmp

      % routine name
      \ifx\indexedroutine\referencedroutine
      \else \par \large \ref{\labeltext}\ \textbf{\texttt{\referencedroutine}}
      \fi
      { \RoutineRefStyle %\catcode`\_=13 \def_{\underscore}
        Figure \ref{\labeltext}\ \textbf{\texttt{\indexedroutine}}
      }

      % routine display
      {\footnotesize
       \def\verbatim@startline{\verbatim@line{\leavevmode\relax}}
        \def\mpifam{mpi}
        \ifx\mpifam\routinefam
            \edef\tmp{\lowercase{\def\noexpand\standardroutine{\referencedroutine}}}\tmp
            \IfFileExists
                {standard/\standardroutine.tex}
                {\input{standard/\standardroutine}}
                {% if no standard file, then maybe handwritten file
                  \IfFileExists
                      {mpireference/\referencedroutine.tex}
                      {\verbatiminput{mpireference/\referencedroutine}}{}}
            \IfFileExists
                  {mplreference/\standardroutine.tex}
                  {MPL:\\ \verbatiminput{mplreference/\standardroutine.tex}}{}
            \IfFileExists
                {standardp/\standardroutine.tex}
                {Python:\\ \input{standardp/\standardroutine.tex}}
                {% no python reference from cython definition
                 \IfFileExists
                     {pyreference/\referencedroutine}
                     {Python:\\ \verbatiminput{pyreference/\referencedroutine.tex}}{}
                }
        \else
            \edef\tmp{\noexpand\verbatiminput{\referencedroutine}}\tmp
        \fi
      }
    \end{manpage}%
  }
}

\newcommand\pylist[1]{Class #1:\verbatiminput{standardp/class_#1.tex}}

\def\boldtt#1{\textbf{\texttt{#1}}}

\newcommand{\indexterm}[1]{\emph{#1}\index{#1}}
\def\indextermdef#1{\emph{#1}\index{#1|textbf}}
{ \catcode`\_=13
\gdef\tttoindex{%\tracingmacros=2
  \edef\tmp{%
    \noexpand\n{\indexedtt}%
    \noexpand\index{\indexedtt@{\catcode95=12 \noexpand\texttt{\indexedtt}}}%
  }%
  \tmp
  \egroup
}
% used for unix, environment variables, stuff
\gdef\indextermtt#{\bgroup \catcode`\_=12 % 13 \def_{\char95\discretionary{}{}{}}
  \catcode`\>=12 \catcode`\<=12
  \catcode`\&=12 \catcode`\^=12 \catcode`\~=12 \def\\{\char`\\}\relax
  \tt \afterassignment\tttoindex\edef\indexedtt}
}
\let\indextermttdef\indextermtt
\let\indexcommand     \indextermtt
\let\indextermunix    \indextermtt
\let\indextermfunction\indextermtt

% -*- latex -*-
%%%%%%%%%%%%%%%%%%%%%%%%%%%%%%%%
%%%%%%%%%%%%%%%%%%%%%%%%%%%%%%%%
%%
%% This text file is part of the source of 
%% `Parallel Computing'
%% by Victor Eijkhout, copyright 2012-2022
%%
%% MPI API file for MPI_Igatherv
%%
%% THIS FILE IS AUTO-GENERATED
%%
%%%%%%%%%%%%%%%%%%%%%%%%%%%%%%%%
%%%%%%%%%%%%%%%%%%%%%%%%%%%%%%%%

\begingroup
\ttfamily\bfseries
\catcode`\_=12
\begin{tabular}{
l       % name
l       % param name
p{\mpiparmtextsize}  % explanation
p{.7in} % ctype
p{.9in} % ftype
l       % inout
}
\toprule
\mdseries\textrm{Name}&
\mdseries\textrm{Param name}&
\mdseries\textrm{Explanation}&
\mdseries\textrm{C type}&
\mdseries\textrm{F type}&
\mdseries\textrm{inout}\\
\midrule
\hbox to 18pt{MPI_Igatherv (\hss} \\
 & sendbuf & starting address of send buffer & const void* & TYPE\discretionary{}{\kern10pt}{}(*), DIMENSION\discretionary{}{\kern10pt}{}(..)  & IN \\ & sendcount & number of elements in send buffer & int & INTEGER  & IN \\ & sendtype & datatype of send buffer elements & MPI_Datatype & TYPE\discretionary{}{\kern10pt}{}(MPI_Datatype)  & IN \\ & recvbuf & address of receive buffer & void* & TYPE\discretionary{}{\kern10pt}{}(*), DIMENSION\discretionary{}{\kern10pt}{}(..)  & OUT \\ & recvcounts & non-negative integer array (of length group size) containing the number of elements that are received from each process & const int[] & INTEGER\discretionary{}{\kern10pt}{}(*)  & IN \\ & displs & integer array (of length group size). Entry \mpicode{i} specifies the displacement relative to \mpiarg{recvbuf} at which to place the incoming data from process \mpicode{i} & const int[] & INTEGER\discretionary{}{\kern10pt}{}(*)  & IN \\ & recvtype & datatype of recv buffer elements & MPI_Datatype & TYPE\discretionary{}{\kern10pt}{}(MPI_Datatype)  & IN \\ & root & rank of receiving process & int & INTEGER  & IN \\ & comm & communicator & MPI_Comm & TYPE\discretionary{}{\kern10pt}{}(MPI_Comm)  & IN \\ & request & communication request & MPI_Request* & TYPE\discretionary{}{\kern10pt}{}(MPI_Request)  & OUT \\
&)\\

\bottomrule
\end{tabular}
\endgroup


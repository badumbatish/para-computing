% -*- latex -*-
%%%%%%%%%%%%%%%%%%%%%%%%%%%%%%%%%%%%%%%%%%%%%%%%%%%%%%%%%%%%%%%%
%%%%%%%%%%%%%%%%%%%%%%%%%%%%%%%%%%%%%%%%%%%%%%%%%%%%%%%%%%%%%%%%
%%%%
%%%% This text file is part of the source of 
%%%% `Parallel Computing'
%%%% by Victor Eijkhout, copyright 2012-2021
%%%%
%%%% MPI API file
%%%%
%%%% THIS FILE IS AUTO-GENERATED
%%%%
%%%%%%%%%%%%%%%%%%%%%%%%%%%%%%%%%%%%%%%%%%%%%%%%%%%%%%%%%%%%%%%%
%%%%%%%%%%%%%%%%%%%%%%%%%%%%%%%%%%%%%%%%%%%%%%%%%%%%%%%%%%%%%%%%

\begingroup\tt\catcode`\_=12
\begin{tabular}{lllll}
\toprule
\textrm{Name}&\textrm{Param name}&\textrm{C type}&\textrm{F type}&\textrm{inout}\\
\midrule
\hbox to 20pt{MPI_T_pvar_get_info (\hss} \\
&pvar_index&int&&in\\ [-3pt]
&\hbox to 0pt{\footnotesize\sl index of the performance variable to be queried between $0$ and $num_pvar-1$\hss}\\
&name&char*&CHARACTER&out\\ [-3pt]
&\hbox to 0pt{\footnotesize\sl buffer to return the string containing the name of the performance variable\hss}\\
&name_len&int*&INTEGER&inout\\ [-3pt]
&\hbox to 0pt{\footnotesize\sl length of the string and/or buffer for name\hss}\\
&verbosity&int*&&out\\ [-3pt]
&\hbox to 0pt{\footnotesize\sl verbosity level of this variable\hss}\\
&var_class&int*&&out\\ [-3pt]
&\hbox to 0pt{\footnotesize\sl class of performance variable\hss}\\
&datatype&MPI_Datatype*&TYPE(MPI_Datatype)&out\\ [-3pt]
&\hbox to 0pt{\footnotesize\sl MPI datatype of the information stored in the performance variable\hss}\\
&enumtype&MPI_T_enum*&&out\\ [-3pt]
&\hbox to 0pt{\footnotesize\sl optional descriptor for enumeration information\hss}\\
&desc&char*&CHARACTER&out\\ [-3pt]
&\hbox to 0pt{\footnotesize\sl buffer to return the string containing a description of the performance variable\hss}\\
&desc_len&int*&INTEGER&inout\\ [-3pt]
&\hbox to 0pt{\footnotesize\sl length of the string and/or buffer for desc\hss}\\
&bind&int*&&out\\ [-3pt]
&\hbox to 0pt{\footnotesize\sl type of MPI object to which this variable must be bound\hss}\\
&readonly&int*&&out\\ [-3pt]
&\hbox to 0pt{\footnotesize\sl flag indicating whether the variable can be written/reset\hss}\\
&continuous&int*&&out\\ [-3pt]
&\hbox to 0pt{\footnotesize\sl flag indicating whether the variable can be started and stopped or is continuously active\hss}\\
&atomic&int*&&out\\ [-3pt]
&\hbox to 0pt{\footnotesize\sl flag indicating whether the variable can be atomically read and reset\hss}\\
(opt)&ierror&&INTEGER&out\\
)\\
\bottomrule
\end{tabular}
\endgroup


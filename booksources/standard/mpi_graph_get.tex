% -*- latex -*-
%%%%%%%%%%%%%%%%%%%%%%%%%%%%%%%%
%%%%%%%%%%%%%%%%%%%%%%%%%%%%%%%%
%%
%% This text file is part of the source of 
%% `Parallel Computing'
%% by Victor Eijkhout, copyright 2012-2021
%%
%% MPI API file for MPI_Graph_get
%%
%% THIS FILE IS AUTO-GENERATED
%%
%%%%%%%%%%%%%%%%%%%%%%%%%%%%%%%%
%%%%%%%%%%%%%%%%%%%%%%%%%%%%%%%%

\begingroup
\ttfamily\bfseries
\catcode`\_=12
\begin{tabular}{lllll}
\toprule
\mdseries\textrm{Name}&
\mdseries\textrm{Param name}&
\mdseries\textrm{C type}&
\mdseries\textrm{F type}&
\mdseries\textrm{inout}\\
\midrule
\hbox to 20pt{MPI_Graph_get (\hss} \\
 & comm & MPI_Comm & TYPE(MPI_Comm)  \\&\hbox to 0pt{\footnotesize\sl communicator with graph structure\hss} \\ 
 & maxindex & int & INTEGER  \\&\hbox to 0pt{\footnotesize\sl length of vector \mpiarg{index} in the calling program\hss} \\ 
 & maxedges & int & INTEGER  \\&\hbox to 0pt{\footnotesize\sl length of vector \mpiarg{edges} in the calling program\hss} \\ 
 & index & int & INTEGER  \\&\hbox to 0pt{\footnotesize\sl array of integers containing the graph structure (for details see the definition of \mpifunc{MPI_GRAPH_CREATE})\hss} \\ 
 & edges & int & INTEGER  \\&\hbox to 0pt{\footnotesize\sl array of integers containing the graph structure\hss} \\ 

\bottomrule
\end{tabular}
\endgroup


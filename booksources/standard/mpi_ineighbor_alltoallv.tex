% -*- latex -*-
%%%%%%%%%%%%%%%%%%%%%%%%%%%%%%%%
%%%%%%%%%%%%%%%%%%%%%%%%%%%%%%%%
%%
%% This text file is part of the source of 
%% `Parallel Computing'
%% by Victor Eijkhout, copyright 2012-2021
%%
%% MPI API file for MPI_Ineighbor_alltoallv
%%
%% THIS FILE IS AUTO-GENERATED
%%
%%%%%%%%%%%%%%%%%%%%%%%%%%%%%%%%
%%%%%%%%%%%%%%%%%%%%%%%%%%%%%%%%

\begingroup
\ttfamily\bfseries
\catcode`\_=12
\begin{tabular}{lllll}
\toprule
\mdseries\textrm{Name}&
\mdseries\textrm{Param name}&
\mdseries\textrm{C type}&
\mdseries\textrm{F type}&
\mdseries\textrm{inout}\\
\midrule
\hbox to 20pt{MPI_Ineighbor_alltoallv (\hss} \\
\hbox to 20pt{MPI_Ineighbor_alltoallv_c (\hss} \\
 & sendbuf & const void* & TYPE(*), DIMENSION(..)  & IN \\&\hbox to 0pt{\footnotesize\sl starting address of send buffer\hss} \\ [+3pt] 
 & sendcounts & const int[] & INTEGER(*)  & IN \\ [-3pt]
(c) & & MPI_Count & \hbox to 0pt {INTEGER(KIND=MPI_COUNT_KIND) \hss} \\ [-3pt]
&\hbox to 0pt{\footnotesize\sl non-negative integer array (of length outdegree) specifying the number of elements to send to each neighbor\hss} \\ [+3pt] 
 & sdispls & const int[] & INTEGER(*)  & IN \\ [-3pt]
(c) & & MPI_Aint & \hbox to 0pt {INTEGER(KIND=MPI_ADDRESS_KIND) \hss} \\ [-3pt]
&\hbox to 0pt{\footnotesize\sl integer array (of length outdegree). Entry \mpicode{j} specifies the displacement (relative to \mpiarg{sendbuf}) from which send the outgoing data to neighbor \mpicode{j}\hss} \\ [+3pt] 
 & sendtype & MPI_Datatype & TYPE(MPI_Datatype)  & IN \\&\hbox to 0pt{\footnotesize\sl datatype of send buffer elements\hss} \\ [+3pt] 
 & recvbuf & void* & TYPE(*), DIMENSION(..)  & OUT \\&\hbox to 0pt{\footnotesize\sl starting address of receive buffer\hss} \\ [+3pt] 
 & recvcounts & const int[] & INTEGER(*)  & IN \\ [-3pt]
(c) & & MPI_Count & \hbox to 0pt {INTEGER(KIND=MPI_COUNT_KIND) \hss} \\ [-3pt]
&\hbox to 0pt{\footnotesize\sl non-negative integer array (of length indegree) specifying the number of elements that are received from each neighbor\hss} \\ [+3pt] 
 & rdispls & const int[] & INTEGER(*)  & IN \\ [-3pt]
(c) & & MPI_Aint & \hbox to 0pt {INTEGER(KIND=MPI_ADDRESS_KIND) \hss} \\ [-3pt]
&\hbox to 0pt{\footnotesize\sl integer array (of length indegree). Entry \mpicode{i} specifies the displacement (relative to \mpiarg{recvbuf}) at which to place the incoming data from neighbor \mpicode{i}\hss} \\ [+3pt] 
 & recvtype & MPI_Datatype & TYPE(MPI_Datatype)  & IN \\&\hbox to 0pt{\footnotesize\sl datatype of receive buffer elements\hss} \\ [+3pt] 
 & comm & MPI_Comm & TYPE(MPI_Comm)  & IN \\&\hbox to 0pt{\footnotesize\sl communicator with topology structure\hss} \\ [+3pt] 
 & request & MPI_Request* & TYPE(MPI_Request)  & OUT \\&\hbox to 0pt{\footnotesize\sl communication request\hss} \\ [+3pt] 

&)\\

\bottomrule
\end{tabular}
\endgroup


% -*- latex -*-
%%%%%%%%%%%%%%%%%%%%%%%%%%%%%%%%
%%%%%%%%%%%%%%%%%%%%%%%%%%%%%%%%
%%
%% This text file is part of the source of 
%% `Parallel Computing'
%% by Victor Eijkhout, copyright 2012-2021
%%
%% MPI API file for MPI_Alltoallv_init
%%
%% THIS FILE IS AUTO-GENERATED
%%
%%%%%%%%%%%%%%%%%%%%%%%%%%%%%%%%
%%%%%%%%%%%%%%%%%%%%%%%%%%%%%%%%

\begingroup
\ttfamily\bfseries
\catcode`\_=12
\begin{tabular}{lllll}
\toprule
\mdseries\textrm{Name}&
\mdseries\textrm{Param name}&
\mdseries\textrm{C type}&
\mdseries\textrm{F type}&
\mdseries\textrm{inout}\\
\midrule
\hbox to 20pt{MPI_Alltoallv_init (\hss} \\
 & sendbuf & const void* & TYPE(*), DIMENSION(..)  & IN \\&\hbox to 0pt{\footnotesize\sl starting address of send buffer\hss} \\ [+3pt] 
 & sendcounts & const int[] & INTEGER(*)  & IN \\&\hbox to 0pt{\footnotesize\sl non-negative integer array (of length group size) specifying the number of elements to send to each rank\hss} \\ [+3pt] 
 & sdispls & const int[] & INTEGER(*)  & IN \\&\hbox to 0pt{\footnotesize\sl Integer array (of length group size). Entry \mpicode{j} specifies the displacement (relative to \mpiarg{sendbuf}) from which to take the outgoing data destined for process \mpicode{j}\hss} \\ [+3pt] 
 & sendtype & MPI_Datatype & TYPE(MPI_Datatype)  & IN \\&\hbox to 0pt{\footnotesize\sl datatype of send buffer elements\hss} \\ [+3pt] 
 & recvbuf & void* & TYPE(*), DIMENSION(..)  & OUT \\&\hbox to 0pt{\footnotesize\sl address of receive buffer\hss} \\ [+3pt] 
 & recvcounts & const int[] & INTEGER(*)  & IN \\&\hbox to 0pt{\footnotesize\sl non-negative integer array (of length group size) specifying the number of elements that can be received from each rank\hss} \\ [+3pt] 
 & rdispls & const int[] & INTEGER(*)  & IN \\&\hbox to 0pt{\footnotesize\sl integer array (of length group size). Entry \mpicode{i} specifies the displacement (relative to \mpiarg{recvbuf}) at which to place the incoming data from process \mpicode{i}\hss} \\ [+3pt] 
 & recvtype & MPI_Datatype & TYPE(MPI_Datatype)  & IN \\&\hbox to 0pt{\footnotesize\sl datatype of receive buffer elements\hss} \\ [+3pt] 
 & comm & MPI_Comm & TYPE(MPI_Comm)  & IN \\&\hbox to 0pt{\footnotesize\sl communicator\hss} \\ [+3pt] 
 & info & MPI_Info & TYPE(MPI_Info)  & IN \\&\hbox to 0pt{\footnotesize\sl info argument\hss} \\ [+3pt] 
 & request & MPI_Request* & TYPE(MPI_Request)  & OUT \\&\hbox to 0pt{\footnotesize\sl communication request\hss} \\ [+3pt] 

&)\\

\bottomrule
\end{tabular}
\endgroup


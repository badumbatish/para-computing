% -*- latex -*-
%%%%%%%%%%%%%%%%%%%%%%%%%%%%%%%%%%%%%%%%%%%%%%%%%%%%%%%%%%%%%%%%
%%%%%%%%%%%%%%%%%%%%%%%%%%%%%%%%%%%%%%%%%%%%%%%%%%%%%%%%%%%%%%%%
%%%%
%%%% This text file is part of the source of 
%%%% `Parallel Computing'
%%%% by Victor Eijkhout, copyright 2012-2021
%%%%
%%%% MPI API file
%%%%
%%%% THIS FILE IS AUTO-GENERATED
%%%%
%%%%%%%%%%%%%%%%%%%%%%%%%%%%%%%%%%%%%%%%%%%%%%%%%%%%%%%%%%%%%%%%
%%%%%%%%%%%%%%%%%%%%%%%%%%%%%%%%%%%%%%%%%%%%%%%%%%%%%%%%%%%%%%%%

\begingroup\tt\catcode`\_=12
\begin{tabular}{lllll}
\toprule
\textrm{Name}&\textrm{Param name}&\textrm{C type}&\textrm{F type}&\textrm{inout}\\
\midrule
\hbox to 20pt{MPI_Type_create_f90_integer (\hss} \\
&r&int&INTEGER&in\\ [-3pt]
&\hbox to 0pt{\footnotesize\sl decimal exponent range, i.e., number of decimal digits\hss}\\
&newtype&MPI_Datatype*&TYPE(MPI_Datatype)&out\\ [-3pt]
&\hbox to 0pt{\footnotesize\sl the requested MPI datatype\hss}\\
(opt)&ierror&&INTEGER&out\\
)\\
\bottomrule
\end{tabular}
\endgroup


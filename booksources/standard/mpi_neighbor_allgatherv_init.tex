\begingroup\tt\catcode`\_=12
\begin{tabular}{lllll}
\toprule
\textrm{Name}&\textrm{Param name}&\textrm{C type}&\textrm{F type}&\textrm{inout}\\
\midrule
\hbox to 20pt{mpi_neighbor_allgatherv_init (\hss} \\
&sendbuf&const~void*&TYPE(*), DIMENSION(..)&in\\ [-3pt]
&\hbox to 0pt{\footnotesize\sl starting address of send buffer\hss}\\
&sendcount&int&INTEGER&in\\ [-3pt]
&\hbox to 0pt{\footnotesize\sl number of elements sent to each neighbor\hss}\\
&sendtype&MPI_Datatype&TYPE(MPI_Datatype)&in\\ [-3pt]
&\hbox to 0pt{\footnotesize\sl data type of send buffer elements\hss}\\
&recvbuf&void*&TYPE(*), DIMENSION(..)&out\\ [-3pt]
&\hbox to 0pt{\footnotesize\sl starting address of receive buffer\hss}\\
&recvcounts&const~int[]&INTEGER&in\\&\hbox to 0pt{\footnotesize length: \tt\catcode`\_=12 *\hss}\\ [-3pt]
&\hbox to 0pt{\footnotesize\sl non-negative integer array (of length indegree) containing the number of elements that are received from each neighbor\hss}\\
&displs&const~int[]&INTEGER&in\\&\hbox to 0pt{\footnotesize length: \tt\catcode`\_=12 *\hss}\\ [-3pt]
&\hbox to 0pt{\footnotesize\sl integer array (of length indegree). Entry i specifies the displacement (relative to recvbuf) at which to place the incoming data from neighbor i\hss}\\
&recvtype&MPI_Datatype&TYPE(MPI_Datatype)&in\\ [-3pt]
&\hbox to 0pt{\footnotesize\sl data type of receive buffer elements\hss}\\
&comm&MPI_Comm&TYPE(MPI_Comm)&in\\ [-3pt]
&\hbox to 0pt{\footnotesize\sl communicator with topology structure\hss}\\
&info&MPI_Info&TYPE(MPI_Info)&in\\ [-3pt]
&\hbox to 0pt{\footnotesize\sl info argument\hss}\\
&request&MPI_Request*&TYPE(MPI_Request)&out\\ [-3pt]
&\hbox to 0pt{\footnotesize\sl communication request\hss}\\
(opt)&ierror&&INTEGER&out\\
)\\
\bottomrule
\end{tabular}
\endgroup


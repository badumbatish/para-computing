% -*- latex -*-
%%%%%%%%%%%%%%%%%%%%%%%%%%%%%%%%
%%%%%%%%%%%%%%%%%%%%%%%%%%%%%%%%
%%
%% This text file is part of the source of 
%% `Parallel Computing'
%% by Victor Eijkhout, copyright 2012-2021
%%
%% MPI API file for MPI_Scatterv_init
%%
%% THIS FILE IS AUTO-GENERATED
%%
%%%%%%%%%%%%%%%%%%%%%%%%%%%%%%%%
%%%%%%%%%%%%%%%%%%%%%%%%%%%%%%%%

\begingroup
\ttfamily\bfseries
\catcode`\_=12
\begin{tabular}{lllll}
\toprule
\mdseries\textrm{Name}&
\mdseries\textrm{Param name}&
\mdseries\textrm{C type}&
\mdseries\textrm{F type}&
\mdseries\textrm{inout}\\
\midrule
\hbox to 20pt{MPI_Scatterv_init (\hss} \\
\hbox to 20pt{MPI_Scatterv_init_c (\hss} \\
 & sendbuf & void* & TYPE(*), DIMENSION(..)  & IN \\&\hbox to 0pt{\footnotesize\sl address of send buffer\hss} \\ 
 & sendcounts & int & INTEGER  & IN \\ [-3pt]
(big) & & MPI_Count & \hbox to 0pt {INTEGER(KIND=MPI_COUNT_KIND) \hss} \\ [-3pt]
&\hbox to 0pt{\footnotesize\sl non-negative integer array (of length group size) specifying the number of elements to send to each rank\hss} \\ 
 & displs & int & INTEGER  & IN \\ [-3pt]
(big) & & MPI_Aint & \hbox to 0pt {INTEGER(KIND=MPI_ADDRESS_KIND) \hss} \\ [-3pt]
&\hbox to 0pt{\footnotesize\sl integer array (of length group size). Entry \mpicode{i} specifies the displacement (relative to \mpiarg{sendbuf}) from which to take the outgoing data to process \mpicode{i}\hss} \\ 
 & sendtype & MPI_Datatype & TYPE(MPI_Datatype)  & IN \\&\hbox to 0pt{\footnotesize\sl datatype of send buffer elements\hss} \\ 
 & recvbuf & void* & TYPE(*), DIMENSION(..)  & OUT \\&\hbox to 0pt{\footnotesize\sl address of receive buffer\hss} \\ 
 & recvcount & int & INTEGER  & IN \\ [-3pt]
(big) & & MPI_Count & \hbox to 0pt {INTEGER(KIND=MPI_COUNT_KIND) \hss} \\ [-3pt]
&\hbox to 0pt{\footnotesize\sl number of elements in receive buffer\hss} \\ 
 & recvtype & MPI_Datatype & TYPE(MPI_Datatype)  & IN \\&\hbox to 0pt{\footnotesize\sl datatype of receive buffer elements\hss} \\ 
 & root & int & INTEGER  & IN \\&\hbox to 0pt{\footnotesize\sl rank of sending process\hss} \\ 
 & comm & MPI_Comm & TYPE(MPI_Comm)  & IN \\&\hbox to 0pt{\footnotesize\sl communicator\hss} \\ 
 & info & MPI_Info & TYPE(MPI_Info)  & IN \\&\hbox to 0pt{\footnotesize\sl info argument\hss} \\ 
 & request & MPI_Request* & TYPE(MPI_Request)  & OUT \\&\hbox to 0pt{\footnotesize\sl communication request\hss} \\ 

\bottomrule
\end{tabular}
\endgroup


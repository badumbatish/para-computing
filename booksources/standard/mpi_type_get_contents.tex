% -*- latex -*-
%%%%%%%%%%%%%%%%%%%%%%%%%%%%%%%%
%%%%%%%%%%%%%%%%%%%%%%%%%%%%%%%%
%%
%% This text file is part of the source of 
%% `Parallel Computing'
%% by Victor Eijkhout, copyright 2012-2021
%%
%% MPI API file for MPI_Type_get_contents
%%
%% THIS FILE IS AUTO-GENERATED
%%
%%%%%%%%%%%%%%%%%%%%%%%%%%%%%%%%
%%%%%%%%%%%%%%%%%%%%%%%%%%%%%%%%

\begingroup
\ttfamily\bfseries
\catcode`\_=12
\begin{tabular}{lllll}
\toprule
\mdseries\textrm{Name}&
\mdseries\textrm{Param name}&
\mdseries\textrm{C type}&
\mdseries\textrm{F type}&
\mdseries\textrm{inout}\\
\midrule
\hbox to 20pt{MPI_Type_get_contents (\hss} \\
\hbox to 20pt{MPI_Type_get_contents_c (\hss} \\
 & datatype & MPI_Datatype & TYPE(MPI_Datatype)  & IN \\&\hbox to 0pt{\footnotesize\sl datatype to decode\hss} \\ [+3pt] 
 & max_integers & int & INTEGER  & IN \\ [-3pt]
(c) & & MPI_Count & \hbox to 0pt {INTEGER(KIND=MPI_COUNT_KIND) \hss} \\ [-3pt]
&\hbox to 0pt{\footnotesize\sl number of elements in \mpiarg{array_of_integers}\hss} \\ [+3pt] 
 & max_addresses & int & INTEGER  & IN \\ [-3pt]
(c) & & MPI_Count & \hbox to 0pt {INTEGER(KIND=MPI_COUNT_KIND) \hss} \\ [-3pt]
&\hbox to 0pt{\footnotesize\sl number of elements in \mpiarg{array_of_addresses}\hss} \\ [+3pt] 
 & max_datatypes & int & INTEGER  & IN \\ [-3pt]
(c) & & MPI_Count & \hbox to 0pt {INTEGER(KIND=MPI_COUNT_KIND) \hss} \\ [-3pt]
&\hbox to 0pt{\footnotesize\sl number of elements in \mpiarg{array_of_large_counts}\hss} \\ [+3pt] 
 & array_of_integers & int[] & INTEGER(max_integers)  & IN \\ [-3pt]
(c) & & MPI_Count & \hbox to 0pt {INTEGER(KIND=MPI_COUNT_KIND) \hss} \\ [-3pt]
&\hbox to 0pt{\footnotesize\sl number of elements in \mpiarg{array_of_datatypes}\hss} \\ [+3pt] 
 & array_of_addresses & MPI_Aint[] & INTEGER(KIND=MPI_ADDRESS_KIND)(max_addresses)  & OUT \\ [-3pt]
(c) & & int & \hbox to 0pt {INTEGER \hss} \\ [-3pt]
&\hbox to 0pt{\footnotesize\sl contains integer arguments used in constructing \mpiarg{datatype}\hss} \\ [+3pt] 
 & array_of_datatypes & MPI_Datatype[] & TYPE(MPI_Datatype)(max_datatypes)  & OUT \\ [-3pt]
(c) & & MPI_Aint & \hbox to 0pt {INTEGER(KIND=MPI_ADDRESS_KIND) \hss} \\ [-3pt]
&\hbox to 0pt{\footnotesize\sl contains address arguments used in constructing \mpiarg{datatype}\hss} \\ [+3pt] 

\bottomrule
\end{tabular}
\endgroup


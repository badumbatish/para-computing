% -*- latex -*-
%%%%%%%%%%%%%%%%%%%%%%%%%%%%%%%%%%%%%%%%%%%%%%%%%%%%%%%%%%%%%%%%
%%%%%%%%%%%%%%%%%%%%%%%%%%%%%%%%%%%%%%%%%%%%%%%%%%%%%%%%%%%%%%%%
%%%%
%%%% This text file is part of the source of 
%%%% `Parallel Computing'
%%%% by Victor Eijkhout, copyright 2012-2021
%%%%
%%%% MPI API file
%%%%
%%%% THIS FILE IS AUTO-GENERATED
%%%%
%%%%%%%%%%%%%%%%%%%%%%%%%%%%%%%%%%%%%%%%%%%%%%%%%%%%%%%%%%%%%%%%
%%%%%%%%%%%%%%%%%%%%%%%%%%%%%%%%%%%%%%%%%%%%%%%%%%%%%%%%%%%%%%%%

\begingroup
\ttfamily\bfseries
\catcode`\_=12
\begin{tabular}{lllll}
\toprule
\mdseries\textrm{Name}&
\mdseries\textrm{Param name}&
\mdseries\textrm{C type}&
\mdseries\textrm{F type}&
\mdseries\textrm{inout}\\
\midrule
\hbox to 20pt{MPI_Ineighbor_alltoallw (\hss} \\
&sendbuf&const~void*&TYPE(*), DIMENSION(..)&in\\ [-3pt]
&\hbox to 0pt{\footnotesize\sl starting address of send buffer\hss}\\
&sendcounts&const~int[]&INTEGER&in\\&\hbox to 0pt{
mfamily length: \ttfamily\mdseries\catcode`\_=12 *\hss}\\ [-3pt]
&\hbox to 0pt{\footnotesize\sl non-negative integer array (of length outdegree) specifying the number of elements to send to each neighbor\hss}\\
&sdispls&const~MPI_Aint[]&INTEGER(KIND=MPI_ADDRESS_KIND)&in\\&\hbox to 0pt{
mfamily length: \ttfamily\mdseries\catcode`\_=12 *\hss}\\ [-3pt]
&\hbox to 0pt{\footnotesize\sl integer array (of length outdegree). Entry j specifies the displacement in bytes (relative to sendbuf) from which to take the outgoing data destined for neighbor j\hss}\\
&sendtypes&const~MPI_Datatype[]&TYPE(MPI_Datatype)&in\\&\hbox to 0pt{
mfamily length: \ttfamily\mdseries\catcode`\_=12 *\hss}\\ [-3pt]
&\hbox to 0pt{\footnotesize\sl array of datatypes (of length outdegree). Entry j specifies the type of data to send to neighbor j\hss}\\
&recvbuf&void*&TYPE(*), DIMENSION(..)&out\\ [-3pt]
&\hbox to 0pt{\footnotesize\sl starting address of receive buffer\hss}\\
&recvcounts&const~int[]&INTEGER&in\\&\hbox to 0pt{
mfamily length: \ttfamily\mdseries\catcode`\_=12 *\hss}\\ [-3pt]
&\hbox to 0pt{\footnotesize\sl non-negative integer array (of length indegree) specifying the number of elements that are received from each neighbor\hss}\\
&rdispls&const~MPI_Aint[]&INTEGER(KIND=MPI_ADDRESS_KIND)&in\\&\hbox to 0pt{
mfamily length: \ttfamily\mdseries\catcode`\_=12 *\hss}\\ [-3pt]
&\hbox to 0pt{\footnotesize\sl integer array (of length indegree). Entry i specifies the displacement in bytes (relative to recvbuf) at which to place the incoming data from neighbor i\hss}\\
&recvtypes&const~MPI_Datatype[]&TYPE(MPI_Datatype)&in\\&\hbox to 0pt{
mfamily length: \ttfamily\mdseries\catcode`\_=12 *\hss}\\ [-3pt]
&\hbox to 0pt{\footnotesize\sl array of datatypes (of length indegree). Entry i specifies the type of data received from neighbor i\hss}\\
&comm&MPI_Comm&TYPE(MPI_Comm)&in\\ [-3pt]
&\hbox to 0pt{\footnotesize\sl communicator with topology structure\hss}\\
&request&MPI_Request*&TYPE(MPI_Request)&out\\ [-3pt]
&\hbox to 0pt{\footnotesize\sl communication request\hss}\\
(opt)&ierror&&INTEGER&out\\
)\\
\bottomrule
\end{tabular}
\endgroup


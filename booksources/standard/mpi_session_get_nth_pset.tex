% -*- latex -*-
%%%%%%%%%%%%%%%%%%%%%%%%%%%%%%%%
%%%%%%%%%%%%%%%%%%%%%%%%%%%%%%%%
%%
%% This text file is part of the source of 
%% `Parallel Computing'
%% by Victor Eijkhout, copyright 2012-2021
%%
%% MPI API file for MPI_Session_get_nth_pset
%%
%% THIS FILE IS AUTO-GENERATED
%%
%%%%%%%%%%%%%%%%%%%%%%%%%%%%%%%%
%%%%%%%%%%%%%%%%%%%%%%%%%%%%%%%%

\begingroup
\ttfamily\bfseries
\catcode`\_=12
\begin{tabular}{lllll}
\toprule
\mdseries\textrm{Name}&
\mdseries\textrm{Param name}&
\mdseries\textrm{C type}&
\mdseries\textrm{F type}&
\mdseries\textrm{inout}\\
\midrule
\hbox to 20pt{MPI_Session_get_nth_pset (\hss} \\
 & session & MPI_Session & TYPE(MPI_Session)  \\&\hbox to 0pt{\footnotesize\sl session\hss} \\ [+3pt] 
 & info & MPI_Info & TYPE(MPI_Info)  \\&\hbox to 0pt{\footnotesize\sl info object\hss} \\ [+3pt] 
 & n & int & INTEGER  \\&\hbox to 0pt{\footnotesize\sl index of the desired process set name\hss} \\ [+3pt] 
 & pset_len & int* & INTEGER  \\&\hbox to 0pt{\footnotesize\sl length of the pset\_name argument\hss} \\ [+3pt] 
 & pset_name & char* & CHARACTER  \\&\hbox to 0pt{\footnotesize\sl name of the \mpiarg{n}th process set\hss} \\ [+3pt] 

\bottomrule
\end{tabular}
\endgroup


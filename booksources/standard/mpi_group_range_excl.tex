% -*- latex -*-
%%%%%%%%%%%%%%%%%%%%%%%%%%%%%%%%
%%%%%%%%%%%%%%%%%%%%%%%%%%%%%%%%
%%
%% This text file is part of the source of 
%% `Parallel Computing'
%% by Victor Eijkhout, copyright 2012-2021
%%
%% MPI API file for MPI_Group_range_excl
%%
%% THIS FILE IS AUTO-GENERATED
%%
%%%%%%%%%%%%%%%%%%%%%%%%%%%%%%%%
%%%%%%%%%%%%%%%%%%%%%%%%%%%%%%%%

\begingroup
\ttfamily\bfseries
\catcode`\_=12
\begin{tabular}{lllll}
\toprule
\mdseries\textrm{Name}&
\mdseries\textrm{Param name}&
\mdseries\textrm{C type}&
\mdseries\textrm{F type}&
\mdseries\textrm{inout}\\
\midrule
\hbox to 20pt{MPI_Group_range_excl (\hss} \\
 & group & MPI_Group & TYPE(MPI_Group)  \\&\hbox to 0pt{\footnotesize\sl group\hss} \\ [+3pt] 
 & n & int & INTEGER  \\&\hbox to 0pt{\footnotesize\sl number of triplets in array \mpiarg{ranges}\hss} \\ [+3pt] 
 & ranges & int & INTEGER  \\&\hbox to 0pt{\footnotesize\sl a one-dimensional array of integer triplets, of the form (first rank, last rank, stride) indicating ranks in \mpiarg{group} of processes to be excluded from the output group \mpiarg{newgroup}\hss} \\ [+3pt] 
 & newgroup & MPI_Group* & TYPE(MPI_Group)  \\&\hbox to 0pt{\footnotesize\sl new group derived from above, preserving the order in \mpiarg{group}\hss} \\ [+3pt] 

\bottomrule
\end{tabular}
\endgroup


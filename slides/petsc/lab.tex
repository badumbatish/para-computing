% -*- latex
%%%%%%%%%%%%%%%%%%%%%%%%%%%%%%%%%%%%%%%%%%%%%%%%%%%%%%%%%%%%%%%%
%%%%
%%%% This TeX file is part of the tutorial
%%%% `Introduction to the PETSc library'
%%%% Victor Eijkhout, eijkhout@tacc.utexas.edu
%%%% copyright Victor Eijkhout 2012-6
%%%%
%%%%%%%%%%%%%%%%%%%%%%%%%%%%%%%%%%%%%%%%%%%%%%%%%%%%%%%%%%%%%%%%

\documentclass[11pt]{artikel3}

\usepackage{pslatex,hyperref}
\def\n{\bgroup\catcode`\_=12 \tt \let\next=}

\begin{document}

\title{Lab exercises for the PETSc short course}
\author{Victor Eijkhout -- Texas Advanced Computing Center}
\date{January 2010}
\maketitle 

In the \n{exercises} directory you see four subdirectories, with the
exercises and the solutions, in both C and Fortran. To compile and
run:
\begin{itemize}
\item Once per login session, do \n{module load petsc/3.0.0-debug}.
\item Go into one of the subdirectories, for instance \n{csources}.
\item To compile an exercise, for instance \n{basic}, do
\begin{verbatim}
  make exercise SOURCE=basic
\end{verbatim}
or in the solutions directory
\begin{verbatim}
  make solution SOURCE=basic
\end{verbatim}
\item edit the provided jobscript:
  \begin{itemize}
  \item insert the program name and options
    \item determine the parallelism for the job: \n{1way} for
      sequential; higher for parallel, up to \n{16way}.
  \end{itemize}
\item Submit the script to the queue with \n{qsub jobscript} and use
  \n{watch qstat} to see if you job is waiting (status \n{qw}) or
  running (\n{r}).
\end{itemize}

\section{A simple main program}

Compile and run \n{basic.c} on more than one processor. Check the output.

Look up the command \n{PetscPrintf} and use it. Check the output
again.

Run the executable with \n{-get_total_flop -options_left}. Check the
output. Oops. Look up the man page for \n{PetscInitialize} and correct
the error. It is a good idea to use \n{-options_left} for every run;
you can add this option to a \n{.petscrc} file in the current
directory or your home directory.

\section{Matrix vector operations}

Take a look at \n{mat.c} or \n{mat.F}. Insert a call to
\n{MatSetValue} or \n{MatSetValues} to construct the matrix elements
as a function of \n{i,j}, for instance $1./(i+j+1.)$ (make sure this
is a floating point expression). Run your
code; the \n{MatView} call will give a crude display of the
matrix. Remove it after you have seen it in action; it is only useful
for small matrices.

Next, create a vector of the same size as the matrix, and insert some
values, for instance using \n{VecSet} or \n{VecSetValues}. 
Use \n{MatMult} to multiply the matrix and the vector. Run your code
with the option \n{-malloc_dump}. Is PETSc complaining about something?
What can you do about it? It is a good idea to use this option on
every run.

There is a block of code immediately after the matrix construction
that is commented out. Activate it, run your code, and see that a
binary file is created. Now write the mirroring code that will read
that file into a matrix if no \n{-n} option is given. See the man page
for \n{MatView} to discover what call to use.

\section{Commandline options, matrix construction, solver
  experimentation}

Take the file \n{ksp.c} or \n{ksp.F}. Compile it and run it. The
program has an argument for setting the matrix size:
\begin{verbatim}
  ./ksp -n 25
\end{verbatim}

First experiment with commandline options for changing the solver.
\begin{itemize}
\item Use \verb+-ksp_monitor+ to view the convergence behaviour. If
  the code reports non-convergence, find the code in
  \verb+$PETSC_ARCH/include/petscksp.h+.
\item Set \verb+-pc_type lu+. How fast does the iterative method
  converge?
\item Set \verb+-pc_type ilu+ and run your code in parallel. Do you
  understand the error message?
\item Run your code with \n{-pc_type ilu -ksp_view}. Study the output.
\end{itemize}

Now make the following changes to the source:
\begin{enumerate}
\item Use \n{PetscOptionsGetReal} to accept an optional commandline
  argument:
\begin{verbatim}
  ./ksp -junk .25
\end{verbatim}
\item Add this value to the last \n{n} rows of the matrix in locations
  $(i,i-3)$. The place to do this is indicated in the source.

  Just setting the matrix elements will lead to a runtime error. From
  the error, can you figure out what is missing in your code? You can
  also look at the file \n{fivepoint.c}: what commands follow the
  setting of the matrix elements?
\item Try increasing values of this parameter, using \n{-pc_type
  ilu}. At some point the iterative method will break down. Study the
  codes that PETSc reports. Find an iterative method that will work:
  the problem here is the choice of \n{KSP} rather than the
  preconditioner, so find a suitable value of \n{-ksp_type}.
\end{enumerate}

\section{Shell matrices}

In this section you will explore the PETSc `shell' matrices: matrices
that are not stored in a data structure, but that are defined by a
routine that computes their action. Take the file \n{shell.c} or
\n{shell.F}. After the main program, a routine \n{mymatmult} is
declared, which is attached by \n{MatShellSetOperation} to the matrix
\n{A} as the means of computing the product \n{MatMult(A,in,out)}, for
instance inside an iterative method.

In addition to the shell matrix \n{A}, the code also creates a
traditional matrix \n{AA}. Your assignment is to make it so that
\n{mymatmult} computes the product $y\leftarrow A^tAx$.

In C, use \n{MatShellSetContext} to attach \n{AA} to \n{A} and
\n{MatShellGetContext} to retrieve it again for use; in Fortran use a
common block (or a module) to store \n{AA}.

The code uses a preconditioner \n{PCNONE}. What happens if you run it
with option \n{-pc_type jacobi}?

\end{document}

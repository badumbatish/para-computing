% -*- latex
%%%%%%%%%%%%%%%%%%%%%%%%%%%%%%%%%%%%%%%%%%%%%%%%%%%%%%%%%%%%%%%%
%%%%
%%%% This TeX file is part of the tutorial
%%%% `Introduction to the PETSc library'
%%%% by Victor Eijkhout, eijkhout@tacc.utexas.edu
%%%%
%%%% copyright Victor Eijkhout 2012-2022
%%%%
%%%%%%%%%%%%%%%%%%%%%%%%%%%%%%%%%%%%%%%%%%%%%%%%%%%%%%%%%%%%%%%%

\sectionframe{Getting started}

\begin{numberedframe}{Include files}
C:
\lstset{language=C}
\begin{lstlisting}
#include "petsc.h"
int main(int argc,char **argv)
\end{lstlisting}
Fortran:
\lstset{language=Fortran}
\begin{lstlisting}
program basic
#include <petsc/finclude/petsc.h>
  use petsc
  implicit none
\end{lstlisting}
Python:
\lstset{language=Python}
\begin{lstlisting}
from petsc4py import PETSc
\end{lstlisting}
\end{numberedframe}

\lstset{language=C}
\begin{numberedframe}{Variable declarations, C}
\begin{lstlisting}
  KSP            solver;
  Mat            A;
  Vec            x,y;
  PetscInt       n = 20;
  PetscScalar    v;
  PetscReal      nrm;
\end{lstlisting}
Note Scalar vs Real
\end{numberedframe}

\lstset{language=Fortran}
\begin{numberedframe}{Variable declarations, F}
\begin{lstlisting}
  KSP            :: solver
  Mat            :: A
  Vec            :: x,y
  PetscInt       :: j(3)
  PetscScalar    :: mv
  PetscReal      :: nrm
\end{lstlisting}
Much like in C
\end{numberedframe}

\lstset{language=C}

\begin{numberedframe}{Library setup, C}
\cverbatimsnippet{petscinit}
Can replace \lstinline{MPI_Init}

General: Every routine has an error return. Catch that value!
\end{numberedframe}

\lstset{language=Fortran}
\begin{numberedframe}{Library setup, F}
\begin{lstlisting}
  call PetscInitialize(PETSC_NULL_CHARACTER,ierr)
      CHKERRQ(ier)
// all the petsc work
  call PetscFinalize(ierr); CHKERRQ(ier)
\end{lstlisting}
Error code is now final parameter. This holds for every PETSc routine.
\end{numberedframe}

\lstset{language=C}

\begin{numberedframe}{A word about datatypes}
PETSc programs can not mix single and double precision, nor real/complex:\\
\lstinline{PetscScalar} is single/double/complex depending on the installation.\\
\lstinline{PetscReal} is always real, even in complex installations.

Similarly, \lstinline{PetscInt} is 32/64 bit depending.

Other scalar data types: \lstinline{PetscBool}, \lstinline{PetscErrorCode}

\begin{taccnote}
\begin{verbatim}
module spider petsc
module avail petsc

module load petsc/3.16-i64 # et cetera
\end{verbatim}
\end{taccnote}

\end{numberedframe}

\begin{numberedframe}{Debug and production}
While you are developing your code:
\begin{verbatim}
module load petsc/3.16-debug
# or 3.16-complexdebug, i64debug, rtxdebug &c
\end{verbatim}
This does bounds tests and other time-wasting error checking.

Production:
\begin{verbatim}
module load petsc/3.16
\end{verbatim}
This will just bomb if your program is not correct.

Every petsc configuration is available as debug and non-debug.
\end{numberedframe}

\begin{exerciseframe}[hello]
Look up the function \lstinline{PetscPrintf} and print a message\\
`This program runs on 27 processors'\\
from process zero.

\begin{itemize}
\item Start with the template code \n{hello.c}/\n{hello.F}
\item (or see slide~\ref{sl:ranksize})
\item Compile with \n{make hello}
\item Part two: use \lstinline{PetscSynchronizedPrintf}
\end{itemize}
\end{exerciseframe}

\petscroutineslide{PetscPrintf}

\begin{numberedframe}{About routine prototypes: C/C++}
  \label{sec:protos}
Prototype:
\begin{lstlisting}
PetscErrorCode VecCreate(MPI_Comm comm,Vec *v);
\end{lstlisting}
Use:
\begin{lstlisting}
PetscErrorCode ierr;
MPI_Comm comm = MPI_COMM_WORLD;
Vec v;
ierr = VecCreate( comm,&vec ); CHKERRQ(ierr).
\end{lstlisting}
(always good idea to catch that error code)
\end{numberedframe}

\lstset{language=Fortran}
\begin{numberedframe}{About routine prototypes: Fortran}
\begin{multicols}{2}
\footnotesize
Prototype
\begin{lstlisting}
Subroutine VecCreate
   ( comm,v,ierr )
Type(MPI_Comm) :: comm
Vec            :: v
PetscErrorCode :: ierr
\end{lstlisting}
Use:
\begin{lstlisting}
Type(MPI_Comm) :: &
    comm = MPI_COMM_WORLD
Vec            :: v
PetscErrorCode :: ierr

call VecCreate(comm,v,ierr)
\end{lstlisting}
\columnbreak
\begin{itemize}
\item Final parameter always error parameter. Do not forget!
\item MPI types are of often \lstinline{Type(MPI_Comm)} and such,
\item PETSc datatypes are handled through the preprocessor.
\end{itemize}
\hbox{}\vfill\hbox{}
\end{multicols}
\end{numberedframe}

\lstset{language=Python}
\begin{numberedframe}{About routine prototypes: Python}
Prototype:
\begin{lstlisting}
# object method
PETSc.Mat.setSizes(self, size, bsize=None)
# class method
PETSc.Sys.Print(type cls, *args, **kwargs)
\end{lstlisting}
Use:
\begin{lstlisting}
from petsc4py import PETSc
PETSc.Sys.Print("detecting n option")
A = PETSc.Mat().create(comm=comm)
A.setSizes( ( (None,matrix_size), (None,matrix_size) ) )
\end{lstlisting}
\end{numberedframe}

\lstset{language=C}
\begin{details}
\begin{numberedframe}{Note to self}
\begin{lstlisting}
  PetscInitialize
    (&argc,&args,0,"Usage: prog -o1 v1 -o2 v2\n");
\end{lstlisting}
run as
\begin{verbatim}
  ./program -help
\end{verbatim}
This displays the usage note, plus all available petsc options.

Not available in Fortran
\end{numberedframe}

\begin{numberedframe}{Routine start/end, C}
Debugging support:
\begin{lstlisting}
PetscFunctionBeginUser;
// all statements
PetscFunctionReturn(0);
\end{lstlisting}
leads to informative tracebacks.

(Only in C, not in Fortran)
\end{numberedframe}

\begin{numberedframe}{Example: function with error}
\cverbatimsnippet{petscbacktrace}
\end{numberedframe}

\begin{numberedframe}{Example: error traceback}
\footnotesize
\begin{verbatim}
[0]PETSC ERROR: We cannot go on like this
[0]PETSC ERROR: See https://www.mcs.anl.gov/petsc/documentation/faq.html for trouble shooting.
[0]PETSC ERROR: Petsc Release Version 3.12.2, Nov, 22, 2019
[0]PETSC ERROR: backtrace on a [computer name]
[0]PETSC ERROR: Configure options [all options]
[0]PETSC ERROR: #1 this_function_bombs() line 20 in backtrace.c
[0]PETSC ERROR: #2 main() line 30 in backtrace.c
\end{verbatim}
\end{numberedframe}

\begin{exerciseframe}[root]
Start with \n{root.c}. Write a function that computes a square root,
or displays an error on negative input:
Look up the definition of \indexpetscshow{SETERRQ1}.
\begin{lstlisting}
  x = 1.5; ierr = square_root(x,&rootx); CHKERRQ(ierr);
  PetscPrintf(PETSC_COMM_WORLD,"Root of %f is %f\n",x,rootx);
  x = -2.6; ierr = square_root(x,&rootx); CHKERRQ(ierr);
  PetscPrintf(PETSC_COMM_WORLD,"Root of %f is %f\n",x,rootx);
\end{lstlisting}
This should give as output:

\begingroup \tiny
\begin{verbatim}
Root of 1.500000 is 1.224745
[0]PETSC ERROR: ----- Error Message ----------------------------------------------
[0]PETSC ERROR: Cannot compute the root of -2.600000
[...]
[0]PETSC ERROR: #1 square_root() line 23 in root.c
[0]PETSC ERROR: #2 main() line 39 in root.c
\end{verbatim}
\endgroup
Fortran: you need to set the \n{ierr} value yourself.
\end{exerciseframe}

\begin{numberedframe}{Commandline arguments, C}
(I'm leaving out the \lstinline{CHKERRQ(ierr)} in the examples,\\
but do use this in actual code)
\begin{lstlisting}
ierr = PetscOptionsGetInt
    (PETSC_NULL,PETSC_NULL,"-n",&n,&flag); CHKERRQ(ierr);
ierr = PetscPrintf
    (comm,"Input parameter: %d\n",n); CHKERRQ(ierr);
\end{lstlisting}
Read commandline argument, print out from processor zero;\\
flag can be \lstinline{PETSC_NULL} if not wanted
\end{numberedframe}

\lstset{language=Fortran}
\begin{numberedframe}{Program parameters, F}
\begin{lstlisting}
  character*80      msg
  call PetscOptionsGetInt(
    PETSC_NULL_OPTIONS, PETSC_NULL_CHARACTER, &
      "-n",n,PETSC_NULL_BOOL,ierr)
  write(msg,10) n
 10 format("Input parameter:",i5)
  call PetscPrintf(PETSC_COMM_WORLD,msg,ierr)
\end{lstlisting}
Less elegant than \lstinline{PetscPrintf} in C

Note the \lstinline{PETSC_NULL_XXX}: Fortran has strict type checking.
\end{numberedframe}

\lstset{language=C}

\end{details}

\lstset{language=Python}
\begin{numberedframe}{Program parameters, Python}
\begin{lstlisting}
nlocal = PETSc.Options().getInt("n",10)
\end{lstlisting}
\end{numberedframe}
\lstset{language=C}

\endinput

\begin{numberedframe}{, C}
\begin{verbatim}
\end{verbatim}
\end{numberedframe}

\begin{numberedframe}{, F}
\begin{verbatim}
\end{verbatim}
\end{numberedframe}


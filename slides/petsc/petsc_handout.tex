% -*- latex -*-
%%%%%%%%%%%%%%%%%%%%%%%%%%%%%%%%%%%%%%%%%%%%%%%%%%%%%%%%%%%%%%%%
%%%%
%%%% This TeX file is part of the tutorial
%%%% `Introduction to the PETSc library'
%%%% by Victor Eijkhout, eijkhout@tacc.utexas.edu
%%%%
%%%% copyright Victor Eijkhout 2012-2020
%%%%
%%%%%%%%%%%%%%%%%%%%%%%%%%%%%%%%%%%%%%%%%%%%%%%%%%%%%%%%%%%%%%%%

\documentclass[11pt]{artikel3}

\usepackage{pslatex,hyperref}

\def\n{\bgroup\catcode`\_=12 \catcode`\~=12 \tt \let\next=}

%% \addtolength\textwidth{1in}
%% \addtolength\textheight{-3in}
%% \addtolength\oddsidemargin{-.5in}

\begin{document}
\title{PETSc course\\ Instructions for the lab exercises}
\author{}\date{}\maketitle

\section*{Session setup}

\begin{itemize}
\item Log in to your account on stampede2.
\item If you program in C/C++/Fortran:
\begin{verbatim}
module load petsc/3.12-debug
\end{verbatim}
\item If you program in Python,
  first load
\begin{verbatim}
module load python2
\end{verbatim}
then set the following environment variables:
\begin{verbatim}
export PETSC_DIR=/work/00434/eijkhout/petsc/petsc-3.12.3
export PETSC_ARCH=skx-intel-debug4py
export PYTHONPATH=${PETSC_DIR}/${PETSC_ARCH}/lib:${PYTHONPATH}
\end{verbatim}
\end{itemize}

\section*{Download materials}

\begin{itemize}
\item Download the course materials by
\begin{verbatim}
wget http://tinyurl.com/tacc-petsc-2020
mv tacc-petsc-2020 tacc-petsc-2020.tgz
tar fxz tacc-petsc-2020.tgz
\end{verbatim}
or (on stampede2):
\begin{verbatim}
cp ~train00/petsc_course_2020.tgz .
tar fxz petsc_course.tgz
\end{verbatim}
Note the dot at the end of the copy command!
\item You should now have a  directory \n{petsc_course} Go there.
\item You will find: 
  \begin{itemize}
  \item four source directories: exercises and solutions for C,
    Fortran, Python
  \item files with the lecture slides and the instructions for the
    exercises.
  \end{itemize}
\end{itemize}

\section*{Interactive development}

\begin{itemize}
\item Set up an interactive session:
\begin{verbatim}
idev -t 8:0:0 -N 1 -n 20
\end{verbatim}
This will give you one node, with 20~cores, for 8~hours which is enough for this
course. If it prompts you that you can use a reservation, accept.
\item Type \n{make} to see what make targets are available. In particular, this
  gives you a list of the exercises you can make.
\item Type \n{make exercisename} to compile it (the exercise name
  will be in the course slides)
\item Run with:
\begin{verbatim}
ibrun exercisename
# or
ibrun python2 exercisename.py
\end{verbatim}
\end{itemize}

\end{document}


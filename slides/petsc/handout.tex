\documentclass[11pt]{artikel3}

\usepackage{pslatex,hyperref}

\def\n{\bgroup\catcode`\_=12 \catcode`\~=12 \tt \let\next=}

\addtolength\textwidth{1in}
\addtolength\textheight{-3in}
\addtolength\oddsidemargin{-.5in}

\begin{document}
\title{RWTH Regional School\\ Instructions for the lab exercises}
\author{}\date{}\maketitle

\section*{Download materials}

\begin{itemize}
\item Download the course materials by
  \n{wget http://tinyurl.com/ISTC-petsc-tutorial} You do not have to do
  this in an empty directory.
\item Expand the file you have just downloaded: \n{tar fxz
  petsc_tutorial_files.tgz} This creates a directory
  \n{petsc_tutorial_files} Go there.
\item You will find: 
  \begin{itemize}
  \item four source directories: exercises and solutions for C and
    Fortran,
  \item files with the lecture slides and the instructions for the
    exercises, and this handout.
  \end{itemize}
\item There is also a job file to submit parallel runs. However, we
  will use a different way of running the exercises.
\end{itemize}

\section*{Labs}

{\it Problems with compiling:} make sure the environment variables
\n{PETSC_DIR} and \n{PETSC_ARCH} are properly set.

{\it Problems with running:} did you use some variant of mpirun? On
your own machine, you
can use the one in \n{\${PETSC_DIR}/\${PETSC_ARCH}/bin}

\subsection*{Lab1: \n{basic.c} / \n{basic.F90}}

Make the example program with the command \n{make basic} and run it in
parallel. You should get something like this:
\begin{verbatim}
Hello, I'm processor 0 and I have just initialised Petsc!
Hello, I'm processor 1 and I have just initialised Petsc!
Hello, I'm processor 2 and I have just initialised Petsc!
Hello, I'm processor 3 and I have just initialised Petsc!
\end{verbatim}
Look up the manpage for \n{PetscSynchronizedPrintf}, and alter the
program so that only processor zero produces output; it should report
how many processes there are in total:
\begin{verbatim}
Processor 0 out of 4 reporting for duty!
\end{verbatim}
You may need to include \n{PetscSynchronizedFlush} in your code.

\subsection*{Lab2: \n{mat.c} / \n{mat.F90}}

The example program contains the skeleton of a matrix construction. By
invoking the program as \n{mat -n 23} you specify the size of the
matrix.  This program needs to be run on a single processor.

\begin{itemize}
\item Insert a test that stops the program if it is run on more than
  one processor: use the \n{SETERRQ} command.
\item Finish the code for inserting elements into the matrix; use any
  simple function of $i$ and~$j$.
\item The intended behaviour of this program is as follows: if it is
  run with the \n{-n} parameter, a matrix is constructed and written
  to file. If the program is subsequently run without the \n{-n}
  parameter, it reads the dump file. Write the \n{else} part of the
  program so that the dump file is read back.
\end{itemize}

\subsection*{Lab3: \n{ksp.c} / \n{ksp.F90}}

This program constructs a simple five-point Laplacian matrix and
solves a linear system with it using an iterative method. Study the
matrix construction, build the executable, and run it in parallel:
\n{mat -n 23} will construct a matrix on a $23\times 23$ domain.

\begin{itemize}
\item Increase the problem size and see that the number of iterations
  increases.
\item Add the option \n{-ksp_view} and see what the maximum number of
  iterations is. Look up the manpage for \n{KSPSetTolerances}, and
  decrease the maximum number of iterations so that it is no longer
  sufficient for convergence at some size. Check the output.
\item The source contains a placeholder for code that inserts a junk
  value somewhere in the matrix. Finish this code, run the program,
  and confirm that the iterative method is likely to diverge.
\item Experiment with commandline options: set a different iterative
  method with \n{-ksp_type}, and once you have convergence, try to
  decrease the number of iterations by choosing a different
  preconditioner with \n{-pc_type}.
\item What happens if you use \n{-pc_type lu} or \n{pc_type ilu} in
  parallel? Search the lecture notes for instructions on how to use a
  parallel direct solver (such as mumps), or a parallel preconditioner
  (such as hypre). 
\item Also try the \n{bjacobi} and \n{aschwarz}
  preconditioners. Experiment with the subsolvers.
\end{itemize}

\end{document}

% -*- latex
%%%%%%%%%%%%%%%%%%%%%%%%%%%%%%%%%%%%%%%%%%%%%%%%%%%%%%%%%%%%%%%%
%%%%
%%%% This TeX file is part of the tutorial
%%%% `Introduction to the PETSc library'
%%%% Victor Eijkhout, eijkhout@tacc.utexas.edu
%%%% copyright Victor Eijkhout 2012-6
%%%%
%%%%%%%%%%%%%%%%%%%%%%%%%%%%%%%%%%%%%%%%%%%%%%%%%%%%%%%%%%%%%%%%

\sectionframe{Making and running PETSc programs}

\subsection{Installation}

\frame[containsverbatim]{\frametitle{Installation}
\begin{verbatim}
cd petsc-3.7
./configure
make
\end{verbatim}
\verb+configure.py --help+ gives

\tiny
\begin{verbatim}
PETSc:
  --prefix=<path>                          : Specifiy location to install PETSc (eg. /usr/local)                                                   current: 
  --with-sudo=sudo                         : Use sudo when installing packages                                                                     current: 
  --with-default-arch=<bool>               : Allow using the last configured arch without setting PETSC_ARCH                                       current: 1
  --PETSC_ARCH                             : The configuration name
  --with-petsc-arch                        : The configuration name
  --PETSC_DIR                              : The root directory of the PETSc installation
  --with-installation-method=<method>      : Method of installation, e.g. tarball, clone,
  et cetera: many more
\end{verbatim}
}

\subsection{Compiling, linking, running to your program}

\frame[containsverbatim]{\frametitle{Compiling}

Petsc compile and link lines are very long!\\
Use PETSc include file with variables and rules:
\begin{verbatim}
include ${PETSC_DIR}/lib/petsc/conf/variables
include ${PETSC_DIR}/lib/petsc/conf/rules
prog : ${OBJS} 
        ${CLINKER} -o prog ${OBJS} ${PETSC_LIB}
\end{verbatim}
My preference:
\begin{verbatim}
include ${PETSC_DIR}/lib/petsc/conf/variables
CFLAGS = ${PETSC_CC_INCLUDES}
FFLAGS = ${PETSC_FC_INCLUDES}
\end{verbatim}
and write your own compile rules
}

\begin{longversion}
\frame[containsverbatim]{
Find out what \n{PETSC_INCLUDE} and \n{PETSC_LIB} are:
\begin{verbatim}
cd $PETSC_DIR ; make getincludedirs getlinklibs
\end{verbatim}
}
\end{longversion}

\frame[containsverbatim]{\frametitle{Environment variables}
\small
  \begin{itemize}
  \item \n{PETSC_DIR} : different for different version numbers
  \item \n{PETSC_ARCH} : for one version, this controls real/complex
    or opt/debug variants
  \end{itemize}
Versions available at TACC: 
\begin{verbatim}
   petsc/3.6-cxx                petsc/3.7-debug
   petsc/3.6-cxxcomplex         petsc/3.7-i64debug
   petsc/3.6-cxxcomplexdebug    petsc/3.7-i64
   petsc/3.6-cxxdebug           petsc/3.7-single
   petsc/3.6-cxxi64debug        petsc/3.7-uni
   petsc/3.6-cxxi64             petsc/3.7-unidebug
   petsc/3.6-debug              petsc/3.7
   petsc/3.6-i64debug
%% module load petsc/3.7-cxxi64debug
%% echo $PETSC_DIR
/home1/apps/intel15/mvapich2_2_1/petsc/3.7/
%% echo $PETSC_ARCH
sandybridge-cxxi64debug
\end{verbatim}
}

\frame[containsverbatim]{\frametitle{Running}
Parallel invocation. On your own machine:
\verb+mpirun -np 3 petscprog <bunch of runtime options>+\\
At TACC: \n{ibrun}

Petsc has lots of runtime options.
\begin{itemize}
\item \n{-log_summary} : give runtime statistics
\item \n{-malloc_dump} \n{-memory_info} : memory statistics
\item \n{-start_in_debugger} : parallel debugging (not on our
  clusters, but very convenient on your laptop)
\item \n{-options_left} : check for mistyped options
\item \n{-ksp_type gmres} (et cetera) : program control
\end{itemize}
more later
}

\frame{\frametitle{Running}
Options can be set:
\begin{itemize}
\item on the commandline
\item in the program by \n{PetscOptionsSetValue}
\item in a file \n{./.petscrc} or \n{\$\{HOME\}/.petscrc}
\end{itemize}
}

\frame[containsverbatim]{\frametitle{Fortran}

  \begin{itemize}
  \item Include files (already mentioned):
\begin{verbatim}
#include <petsc/finclude/petscdef.h>
  use petsc
\end{verbatim}
  \item Separate F90 version of various `Get' routines
\begin{verbatim}
VecGetArrayF90(Vec x,{Scalar, pointer :: xx_v(:)},integer ierr)
\end{verbatim}
\item Null pointers: C~is tolerant for 0 or \n{PETSC_NULL}, Fortran
    use \n{PETSC_NULL_CHARACTER}, \n{PETSC_NULL_INTEGER} et cetera.\\
    Example:
\begin{verbatim}
call PetscOptionsGetInt(PETSC_NULL_CHARACTER,"-name",
    N,flg,ierr)
\end{verbatim}
  \end{itemize}
}

\begin{longversion}
\frame[containsverbatim]{\frametitle{More Fortran}
  \begin{itemize}
  \item Indexing in \n{MatSetValues} and such is always zero-based
  \item Function pointers: routine is assumed to be in the same language
  \end{itemize}
}
\end{longversion}

\begin{frame}[containsverbatim]{Python bindings}
  \url{https://pythonhosted.org/petsc4py/}
\end{frame}

\begin{comment}
\subsection{Source code organization}

\frame[containsverbatim]{\frametitle{Program/subprogram heading}
CPP definition of (sub)program name:\\
will be used in traceback
\begin{verbatim}
#undef __FUNCT__
#define __FUNCT__ "main"
int main(int argc,char **args)
{
  PetscFunctionBegin;
  ...
  PetscFunctionReturn(0);
}
\end{verbatim}
Use this for every subprogram.

Not available in Fortran
}

\frame[containsverbatim]{\frametitle{Petsc function return values}
Every PETSc function returns an error code, zero is success.

C:
\begin{verbatim}
ierr = SomePetscFunction(....); CHKERRQ(ierr);
\end{verbatim}
Fortran:
\begin{verbatim}
call SomePetscFunction(...., ierr )
CHKERRQ(ierr)
\end{verbatim}
}

\frame[containsverbatim]{\frametitle{Petsc initialize / finalize}

One-time initialization, includes MPI if not already initialized:
\begin{verbatim}
  ierr = PetscInitialize(&argc,&args,0,0); CHKERRQ(ierr);
  ....
  ierr = PetscFinalize(); CHKERRQ(ierr);
\end{verbatim}

Fortran:
\begin{verbatim}
  call PetscInitialize(PETSC_NULL_CHARACTER,ierr)
  CHKERRQ(ierr)
  ....
  call PetscFinalize(ierr)
  CHKERRQ(ierr)
\end{verbatim}
}

\frame[containsverbatim]{\frametitle{Note to self}
\begin{verbatim}
  PetscInitialize(&argc,&args,0,"Usage: prog -o1 v1 -o2 v2\n");
\end{verbatim}
run as
\begin{verbatim}
  ./program -help
\end{verbatim}
displays note, plus all available petsc options.

Not available in Fortran
}

\frame[containsverbatim]{\frametitle{Variable declarations}

Everything is an object
\begin{verbatim}
  MPI_Comm comm;
  PetscErrorCode ierr; PetscTruth flag; 
  KSP Solver; Mat A; Vec Rhs,Sol;
  PetscScalar one; PetscInt its; PetscReal norm;
\end{verbatim}

\begin{verbatim}
  PetscErrorCode ierr
  PetscTruth flag
  KSP Solver
  PC Prec
  Mat A
  PetscInt its 
  ....
\end{verbatim}
(note scalar vs real)
}
\end{comment}

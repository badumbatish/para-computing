% -*- latex -*-
%%%%%%%%%%%%%%%%%%%%%%%%%%%%%%%%%%%%%%%%%%%%%%%%%%%%%%%%%%%%%%%%
%%%%
%%%% This TeX file is part of the tutorial
%%%% `Introduction to the PETSc library'
%%%% Victor Eijkhout, eijkhout@tacc.utexas.edu
%%%% copyright Victor Eijkhout 2012-2020
%%%%
%%%%%%%%%%%%%%%%%%%%%%%%%%%%%%%%%%%%%%%%%%%%%%%%%%%%%%%%%%%%%%%%

\sectionframe{Profiling, debugging}
\frame[containsverbatim]{\frametitle{Basic profiling}
%  Most profiling and debugging requires PETSc installation with debug.

  \begin{itemize}
  \item \n{-log_summary} flop counts and timings of all PETSc events
  \item \n{-info} all sorts of information, in particular
\begin{verbatim}
%% mpiexec yourprogram -info | grep malloc
[0] MatAssemblyEnd_SeqAIJ(): 
    Number of mallocs during MatSetValues() is 0
\end{verbatim}
  \item \n{-log_trace} start and end of all events: good for hanging code
  \end{itemize}
}

\frame[containsverbatim]{\frametitle{Log summary: overall}
\small
\begin{verbatim}
                     Max       Max/Min        Avg      Total 
Time (sec):       5.493e-01      1.00006   5.493e-01
Objects:          2.900e+01      1.00000   2.900e+01
Flops:            1.373e+07      1.00000   1.373e+07  2.746e+07
Flops/sec:        2.499e+07      1.00006   2.499e+07  4.998e+07
Memory:           1.936e+06      1.00000              3.871e+06
MPI Messages:     1.040e+02      1.00000   1.040e+02  2.080e+02
MPI Msg Lengths:  4.772e+05      1.00000   4.588e+03  9.544e+05
MPI Reductions:   1.450e+02      1.00000
\end{verbatim}
}

\frame[containsverbatim]{\frametitle{Log summary: details}
\tiny
\begin{verbatim}
                   Max Ratio  Max     Ratio   Max  Ratio Avg len %T %F %M %L %R  %T %F %M %L %R Mflop/s
MatMult              100 1.0 3.4934e-02 1.0 1.28e+08 1.0 8.0e+02  6 32 96 17  0   6 32 96 17  0   255
MatSolve             101 1.0 2.9381e-02 1.0 1.53e+08 1.0 0.0e+00  5 33  0  0  0   5 33  0  0  0   305
MatLUFactorNum         1 1.0 2.0621e-03 1.0 2.18e+07 1.0 0.0e+00  0  0  0  0  0   0  0  0  0  0    43
MatAssemblyBegin       1 1.0 2.8350e-03 1.1 0.00e+00 0.0 1.3e+05  0  0  3 83  1   0  0  3 83  1     0
MatAssemblyEnd         1 1.0 8.8258e-03 1.0 0.00e+00 0.0 4.0e+02  2  0  1  0  3   2  0  1  0  3     0
VecDot               101 1.0 8.3244e-03 1.2 1.43e+08 1.2 0.0e+00  1  7  0  0 35   1  7  0  0 35   243
KSPSetup               2 1.0 1.9123e-02 1.0 0.00e+00 0.0 0.0e+00  3  0  0  0  2   3  0  0  0  2     0
KSPSolve               1 1.0 1.4158e-01 1.0 9.70e+07 1.0 8.0e+02 26100 96 17 92  26100 96 17 92   194
\end{verbatim}
}

\begin{longversion}
\frame[containsverbatim]{\frametitle{User events}
\begin{lstlisting}
#include "petsclog.h"
int USER EVENT;
PetscLogEventRegister(&USER EVENT,"User event name",0);
PetscLogEventBegin(USER EVENT,0,0,0,0);
/* application code segment to monitor */
PetscLogFlops(number of flops for this code segment);
PetscLogEventEnd(USER EVENT,0,0,0,0);
\end{lstlisting}
}

\frame[containsverbatim]{\frametitle{Program stages}
\begin{lstlisting}
PetscLogStagePush(int stage); /* 0 <= stage <= 9 */
PetscLogStagePop();
PetscLogStageRegister(int stage,char *name)
\end{lstlisting}
}
\end{longversion}

\frame[containsverbatim]{\frametitle{Debugging}

  \begin{itemize}
  \item Use of \n{CHKERRQ} and \n{SETERRQ} for catching and generating
    error
  \item Use of \n{PetscMalloc} and \n{PetscFree} to catch memory problems;\\
    \n{CHKMEMQ} for instantaneous memory test (debug mode only)
  \item Better than \n{PetscMalloc}: \n{PetscMalloc1} aligned to \n{PETSC_MEMALIGN}
  \end{itemize}
}

%\frame[containsverbatim]{\frametitle{debugging on lonestar}}

% \frame[containsverbatim]{\frametitle{valgrind}
% mpirun -np 2 valgrind --tool=memcheck ./ex2

% And make sure that you use mpich2 - for this to work.
% }

